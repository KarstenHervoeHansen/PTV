\section{Command reference}
\label{cha:CommandRef}
\subsection{Reference Documents}

\begin{itemize}
\item IEEE 488.2-1987: IEEE Standard Codes, Formats, Protocols, and Common Commands
\item SCPI 1995.0: Standard Commands for Programmable Instruments, Vol I-IV.
\item ``A Beginner's Guide to SCPI'', Barry Epler, HEWLETT-PACKARD PRESS, 1991
\end{itemize}

\subsection{Configuration and Syntax}

\textbf{Control characters:}

\begin{tabular}{|l|l|}
\hline
Control character 	& Function \\ \hline
$<$Ctrl L$>$ 0x0C 	& Local lockout switchover. \\ 
										& Local lockout is \textbf{always} disabled after power-up \\ \hline
0x0A	 							& Terminator, i.e. newline $<$LF$>$ \\ \hline
\end{tabular}

\textbf{PT8604 Multiple parallel Black Burst:}


Multiple parallel Black Burst option is defined as BB2 when programming via RS232 except when requesting the version of the option. In this case a specific command exists.

%\textbf{Buffers}
%
%\begin{itemize}
%\item Receive buffer: 64 bytes
%\item Transmit buffer: 32 bytes
%\end{itemize}
%
%\textbf{Serial Port:}
%
%The 9-pin RS232 connector consists of:
%
%\begin{tabular}{|l|l|l|}
%\hline
%\textbf{Pin:} & \textbf{Name:} & \textbf{Description:} \\ \hline
%\textbf{1}		& DCD & Not used\\ \hline
%\textbf{2}		& RxD & Receiver pin\\ \hline
%\textbf{3}		& TxD & Transmitter pin\\ \hline
%\textbf{4}		& DTR & Not used\\ \hline
%\textbf{5}		& SG & Signal ground\\ \hline
%\textbf{6}		& DSR & Not used\\ \hline
%\textbf{7}		& RTS & Request to send\\ \hline
%\textbf{8}		& CTS & Clear to send\\ \hline
%\textbf{9}		& RI & Not used\\ \hline
%\end{tabular}

\subsection{Commands Summary}

All commands listed consist of both a set- and request-command unless specifically listed in the table as $<$query only$>$ or $<$no query$>$.

\begin{landscape}

\subsubsection{Mandated Commands}

\commandtable
*CLS	& - & & Clear Status Command \\ \hline
*ESE & & & \\ \hline
*ESE? & & & \\ \hline
*ESR? & & & \\ \hline
*IDN? & & & Device Identification Query \\ \hline
*OPC& & & \\ \hline
*OPC?& & & \\ \hline
*RST & & & Reset Command \\ \hline
*SRE& & & \\ \hline
*SRE?& & & \\ \hline
*STB?& & & \\ \hline
*TST?& & & \\ \hline
*WAI& & & \\ \hline
\end{longtable}



\subsubsection{Instrument Commands}

\textbf{DIAGnostic subsystem}

\commandtable
:DISPlay 		& - 					& 										& $<$no query$>$ \\ \hline
:ERRorqueue & 						& 										& 						\\ \hline
\hspace{1em}:RESet 	& - 	&	- 									&	$<$no query$>$ \\ \hline
:ERRorqueue & - 					&	- 									&	$<$query only$>$ \\ \hline
\end{longtable}

\textbf{DISPlay Subsystem}


\commandtable
:CONTrast & $<$0 to 20$>$ $|$ MIN $|$ MAX  & 16 & \\ \hline
\end{longtable}

\textbf{INPut Subsystem}


\commandtable
:GENLock 	& & & \\ \hline
\hspace{1em}:INPut		& A $|$ B $|$ A\_B $|$ SDI $|$ INTernal $|$ INTernal2 & A & \\ \hline
:SYSTem	& PALBurst $|$ NTScburst $|$ SYNC625 $|$ SYNC525 $|$ SDI625 $|$ SDI525 $|$ F358MHz $|$ F443MHz $|$ F5MHz $|$ F10MHz & \hspace{1em}PALBurst & \\ \hline
\hspace{1em}:DElay & $<$Field$>$,$<$Line$>$,$<$HTime$>$ & 0,0,0 & \\ \hline
:GENLock? & - & - & $<$query only$>$ \\ \hline
:SDIGenlock & & & \\ \hline
\hspace{1em}:VERSion? & - & & $<$query only$>$ \\ \hline
\end{longtable}

\textbf{OUTPut Subsystem}



\commandtable
:BB1 & & & \\ \hline
\hspace{1em}:SYSTem & PAL $|$ PAL\_ID $|$ NTSC & PAL & \\ \hline
\hspace{1em}:DELay & $<$Field$>$,$<$Line$>$,$<$HTime$>$ & 0,0,0 & \\ \hline
\hspace{1em}:SCHPhase & $<$179 to +180$>$ & 0 & \\ \hline
\hspace{1em}:VERSion? & - & - & $<$query only$>$ \\ \hline
:BB1? & - & - & $<$query only$>$ \\ \hline


& & & \\ \hline
:BB2 & & & \\ \hline
\hspace{1em}:SYSTem & PAL $|$ PAL\_ID $|$ NTSC & PAL & \\ \hline
\hspace{1em}:DELay & $<$Field$>$,$<$Line$>$,$<$HTime$>$ & 0,0,0 & \\ \hline
\hspace{1em}:SCHPhase & $<$179 to +180$>$ & 0 & \\ \hline
\hspace{1em}:VERSion? & - & - & $<$query only$>$ \\ \hline
:BB2? & - & - & $<$query only$>$ \\ \hline


& & & \\ \hline
:ATPGenerator2 & & & \\ \hline
\hspace{1em}:PATTerrn & $<$pattern\_name$>$ & CBEBu & \\ \hline
\hspace{1em}:MODify 	& OFF $|$ ON & & \\ \hline
\hspace{2em}:APAL 		& OFF $|$ ON & & \\ \hline
\hspace{2em}:PLUGe 		& OFF $|$ ON & & \\ \hline
\hspace{2em}:STAircase10	& OFF $|$ ON & & \\ \hline
\hspace{1em}:TEXT			&	& & \\ \hline
\hspace{2em}STRing1 	& OFF $|$ ON $<$string\_data$>$ & OFF,``ANALOG1'' OFF,``PTV'' & Standard Pattern Complex Pattern\\ \hline
\hspace{2em}STRing2		& OFF $|$ ON $<$string\_data$>$ & OFF,``ANALOG2'' OFF,``PT5230'' & Standard Pattern Complex Pattern\\ \hline
\hspace{2em}STRing3		& OFF $|$ ON $<$string\_data$>$ & OFF,``ANALOG3'' & Standard Pattern\\ \hline
\hspace{2em}STYLe			& STANdard $|$ COMPlex & COMPlex & \\ \hline 
\hspace{2em}CLOCk 		& OFF $|$ TIME $|$ DTIME & OFF & \\ \hline
\hspace{1em}:SYSTem 	& PAL $|$ PAL\_ID $|$ NTSC & PAL & \\ \hline
\hspace{1em}:DELay 		& $<$Field$>$,$<$Line$>$,$<$HTime$>$ & 0,0,0 & \\ \hline
\hspace{1em}:SCHPhase & $<$-179 to 180$>$ & 0 & \\ \hline
\hspace{1em}:VERSion?	& - & - & $<$query only$>$ \\ \hline
:ATPGenerator2? 			& - & - & $<$query only$>$ \\ \hline


& & & \\ \hline
:ATPGenerator5 & & & \\ \hline
\hspace{1em}:PATTerrn & $<$pattern\_name$>$ & CBEBu & \\ \hline
\hspace{1em}:MODify 	& OFF $|$ ON & & \\ \hline
\hspace{2em}:APAL 		& OFF $|$ ON & & \\ \hline
\hspace{2em}:PLUGe 		& OFF $|$ ON & & \\ \hline
\hspace{2em}:STAircase10	& OFF $|$ ON & & \\ \hline
\hspace{1em}:TEXT			&	& & \\ \hline
\hspace{2em}STRing1 	& OFF $|$ ON $<$string\_data$>$ & OFF,``ANALOG1'' OFF,``PTV'' & Standard Pattern Complex Pattern\\ \hline
\hspace{2em}STRing2		& OFF $|$ ON $<$string\_data$>$ & OFF,``ANALOG2'' OFF,``PT5230'' & Standard Pattern Complex Pattern\\ \hline
\hspace{2em}STRing3		& OFF $|$ ON $<$string\_data$>$ & OFF,``ANALOG3'' & Standard Pattern\\ \hline
\hspace{2em}STYLe			& STANdard $|$ COMPlex & COMPlex & \\ \hline 
\hspace{2em}CLOCk 		& OFF $|$ TIME $|$ DTIME & OFF & \\ \hline
\hspace{1em}:SYSTem 	& PAL $|$ PAL\_ID $|$ NTSC & PAL & \\ \hline
\hspace{1em}:DELay 		& $<$Field$>$,$<$Line$>$,$<$HTime$>$ & 0,0,0 & \\ \hline
\hspace{1em}:SCHPhase & $<$-179 to 180$>$ & 0 & \\ \hline
\hspace{1em}:VERSion?	& - & - & $<$query only$>$ \\ \hline
:ATPGenerator5? 			& - & - & $<$query only$>$ \\ \hline


& & & \\ \hline
:STSGenerator2 				&		&		& \\ \hline	
\hspace{1em}:PATTerrn & $<$pattern\_name$>$	& BLACk & 	\\ \hline
\hspace{1em}:SYSTem		& SDI525 $|$ SDI625		& SDI625 & \\ \hline
\hspace{1em}:DELay		& $<$Field$>$,$<$Line$>$,$<$HTime$>$ & 0,0,0 & \\ \hline
\hspace{1em}:EDHinsert& OFF $|$ ON &	OFF & \\ \hline
\hspace{1em}:EMBaudio & OFF $|$ SILence	& OFF & \\ \hline
\hspace{2em}:SIGNal 	& OFF $|$ S1KHZ 	& OFF & \\ \hline
\hspace{2em}:LEVel		& SILence $|$ DB0FS $|$ DB9FS $|$ DB15FS $|$ DB18FS & SILence & \\ \hline
\hspace{1em}:VERSion? & - & - & $<$query only$>$ \\ \hline
:STSGenerator2? 				& - & - & $<$query only$>$ \\ \hline


& & & \\ \hline
:STSGenerator3 				&		&		& \\ \hline	
\hspace{1em}:PATTerrn & $<$pattern\_name$>$	& BLACk & 	\\ \hline
\hspace{1em}:SYSTem		& SDI525 $|$ SDI625		& SDI625 & \\ \hline
\hspace{1em}:DELay		& $<$Field$>$,$<$Line$>$,$<$HTime$>$ & 0,0,0 & \\ \hline
\hspace{1em}:EDHinsert& OFF $|$ ON &	OFF & \\ \hline
\hspace{1em}:EMBaudio & OFF $|$ SILence	& OFF & \\ \hline
\hspace{2em}:SIGNal 	& OFF $|$ S1KHZ 	& OFF & \\ \hline
\hspace{2em}:LEVel		& SILence $|$ DB0FS $|$ DB9FS $|$ DB15FS $|$ DB18FS & SILence & \\ \hline
\hspace{1em}:VERSion? & - & - & $<$query only$>$ \\ \hline
:STSGenerator3 				& - & - & $<$query only$>$ \\ \hline


& & & \\ \hline
:STSGenerator4 				&		&		& \\ \hline	
\hspace{1em}:PATTerrn & $<$pattern\_name$>$	& BLACk & 	\\ \hline
\hspace{1em}:SYSTem		& SDI525 $|$ SDI625		& SDI625 & \\ \hline
\hspace{1em}:DELay		& $<$Field$>$,$<$Line$>$,$<$HTime$>$ & 0,0,0 & \\ \hline
\hspace{1em}:EDHinsert& OFF $|$ ON &	OFF & \\ \hline
\hspace{1em}:EMBaudio & OFF $|$ SILence	& OFF & \\ \hline
\hspace{2em}:SIGNal 	& OFF $|$ S1KHZ 	& OFF & \\ \hline
\hspace{2em}:LEVel		& SILence $|$ DB0FS $|$ DB9FS $|$ DB15FS $|$ DB18FS & SILence & \\ \hline
\hspace{1em}:VERSion? & - & - & $<$query only$>$ \\ \hline
:STSGenerator4 				& - & - & $<$query only$>$ \\ \hline


& & & \\ \hline
:STPGenerator1				&		&		& \\ \hline	
\hspace{1em}:PATTerrn	& $<$pattern\_name$>$ & CBEBu & \\ \hline
\hspace{1em}:MODify		& OFF $|$ ON & & \\ \hline
\hspace{2em}:APAL			& OFF $|$ ON & OFF & \\ \hline
\hspace{2em}:PLUGe		& OFF $|$ ON & OFF & \\ \hline
\hspace{2em}:STAircase10 &OFF $|$ ON & OFF & \\ \hline
\hspace{2em}MOTion		& OFF $|$ ON & OFF & \\ \hline
\hspace{1em}:TEXT			&	&	& \\ \hline
\hspace{2em}STRing1		& OFF $|$ ON $|$ $<$string\_data$>$ & OFF,``DIGITAL1'' OFF,``PTV'' & Standard Pattern Complex Pattern \\ \hline
\hspace{2em}STRing2		& OFF $|$ ON $|$ $<$string\_data$>$ & OFF,``DIGITAL2'' OFF,``PT5230'' & Standard Pattern Complex Pattern \\ \hline
\hspace{2em}STRing3		& OFF $|$ ON $|$ $<$string\_data$>$ & OFF,``DIGITAL3'' & Standard Pattern \\ \hline
\hspace{2em}STYLe			& STANdard $|$ COMPlex & COMPlex & \\ \hline
\hspace{2em}CLOCk			& OFF $|$ TIME $|$ DTIME & OFF & \\ \hline
\hspace{1em}:SYSTem		& SDI525 $|$ SDI625 & SDI625 & \\ \hline
\hspace{1em}:EDHinsert	& OFF $|$ ON	& OFF & \\ \hline
\hspace{1em}:EMBaudio	& OFF $|$ SILence & OFF & \\ \hline
\hspace{2em}:SIGNal		& OFF $|$ S800HZ $|$ S1KHZ $|$ SEBu1KHz $|$ SBBC1KHZ $|$ MEBU1KHZ $|$ M1KHZ $|$ DUAL & OFF & \\ \hline
\hspace{2em}:LEVel		& SILence $|$ DB0FS $|$ DB9FS $|$ DB12FS $|$ DB15FS $|$ DB16FS $|$ DB18FS $|$ DB20FS & SILence & \\ \hline
\hspace{1em}:DELay		& $<$Field$>$,$<$Line$>$,$<$HTime$>$ & 0,0,0 & \\ \hline
\hspace{1em}:VERSion? 	& - & - & $<$query only$>$ \\ \hline
:STPGenerator1?				& - & - & $<$query only$>$ \\ \hline


& & & \\ \hline
:STPGenerator2				&		&		& \\ \hline	
\hspace{1em}:PATTerrn	& $<$pattern\_name$>$ & CBEBu & \\ \hline
\hspace{1em}:MODify		& OFF $|$ ON & & \\ \hline
\hspace{2em}:APAL			& OFF $|$ ON & OFF & \\ \hline
\hspace{2em}:PLUGe		& OFF $|$ ON & OFF & \\ \hline
\hspace{2em}:STAircase10 &OFF $|$ ON & OFF & \\ \hline
\hspace{2em}MOTion		& OFF $|$ ON & OFF & \\ \hline
\hspace{1em}:TEXT			&	&	& \\ \hline
\hspace{2em}STRing1		& OFF $|$ ON $|$ $<$string\_data$>$ & OFF,``DIGITAL1'' OFF,``PTV'' & Standard Pattern Complex Pattern \\ \hline
\hspace{2em}STRing2		& OFF $|$ ON $|$ $<$string\_data$>$ & OFF,``DIGITAL2'' OFF,``PT5230'' & Standard Pattern Complex Pattern \\ \hline
\hspace{2em}STRing3		& OFF $|$ ON $|$ $<$string\_data$>$ & OFF,``DIGITAL3'' & Standard Pattern \\ \hline
\hspace{2em}STYLe			& STANdard $|$ COMPlex & COMPlex & \\ \hline
\hspace{2em}CLOCk			& OFF $|$ TIME $|$ DTIME & OFF & \\ \hline
\hspace{1em}:SYSTem		& SDI525 $|$ SDI625 & SDI625 & \\ \hline
\hspace{1em}:EDHinsert	& OFF $|$ ON	& OFF & \\ \hline
\hspace{1em}:EMBaudio	& OFF $|$ SILence & OFF & \\ \hline
\hspace{2em}:SIGNal		& OFF $|$ S800HZ $|$ S1KHZ $|$ SEBu1KHz $|$ SBBC1KHZ $|$ MEBU1KHZ $|$ M1KHZ $|$ DUAL & OFF & \\ \hline
\hspace{2em}:LEVel		& SILence $|$ DB0FS $|$ DB9FS $|$ DB12FS $|$ DB15FS $|$ DB16FS $|$ DB18FS $|$ DB20FS & SILence & \\ \hline
\hspace{1em}:DELay		& $<$Field$>$,$<$Line$>$,$<$HTime$>$ & 0,0,0 & \\ \hline
\hspace{1em}:VERSion? 	& - & - & $<$query only$>$ \\ \hline
:STPGenerator2?				& - & - & $<$query only$>$ \\ \hline


& & & \\ \hline
:STPGenerator5				&		&		& \\ \hline	
\hspace{1em}:PATTerrn	& $<$pattern\_name$>$ & CBEBu & \\ \hline
\hspace{1em}:MODify		& OFF $|$ ON & & \\ \hline
\hspace{2em}:APAL			& OFF $|$ ON & OFF & \\ \hline
\hspace{2em}:PLUGe		& OFF $|$ ON & OFF & \\ \hline
\hspace{2em}:STAircase10 &OFF $|$ ON & OFF & \\ \hline
\hspace{2em}MOTion		& OFF $|$ ON & OFF & \\ \hline
\hspace{1em}:TEXT			&	&	& \\ \hline
\hspace{2em}STRing1		& OFF $|$ ON $|$ $<$string\_data$>$ & OFF,``DIGITAL1'' OFF,``PTV'' & Standard Pattern Complex Pattern \\ \hline
\hspace{2em}STRing2		& OFF $|$ ON $|$ $<$string\_data$>$ & OFF,``DIGITAL2'' OFF,``PT5230'' & Standard Pattern Complex Pattern \\ \hline
\hspace{2em}STRing3		& OFF $|$ ON $|$ $<$string\_data$>$ & OFF,``DIGITAL3'' & Standard Pattern \\ \hline
\hspace{2em}STYLe			& STANdard $|$ COMPlex & COMPlex & \\ \hline
\hspace{2em}CLOCk			& OFF $|$ TIME $|$ DTIME & OFF & \\ \hline
\hspace{1em}:SYSTem		& SDI525 $|$ SDI625 & SDI625 & \\ \hline
\hspace{1em}:EDHinsert	& OFF $|$ ON	& OFF & \\ \hline
\hspace{1em}:EMBaudio	& OFF $|$ SILence & OFF & \\ \hline
\hspace{2em}:SIGNal		& OFF $|$ S800HZ $|$ S1KHZ $|$ SEBu1KHz $|$ SBBC1KHZ $|$ MEBU1KHZ $|$ M1KHZ $|$ DUAL & OFF & \\ \hline
\hspace{2em}:LEVel		& SILence $|$ DB0FS $|$ DB9FS $|$ DB12FS $|$ DB15FS $|$ DB16FS $|$ DB18FS $|$ DB20FS & SILence & \\ \hline
\hspace{1em}:DELay		& $<$Field$>$,$<$Line$>$,$<$HTime$>$ & 0,0,0 & \\ \hline
\hspace{1em}:VERSion? 	& - & - & $<$query only$>$ \\ \hline
:STPGenerator5?				& - & - & $<$query only$>$ \\ \hline


& & & \\ \hline
:AUDio1								&	&	&	\\ \hline
\hspace{1em}:SIGnal 	& S800Hz $|$ S1kHz $|$ SEBu1kHz $|$ SBBc1kHz $|$ MEBU1kHz $|$ M1kHz $|$ DUAL $|$ F48kHz $|$ WORDclock & S800Hz & \\ \hline
\hspace{1em}:LEVel 		& SILence $|$ DB0FS $|$ DB9FS $|$ DB12FS $|$ DB15FS $|$ DB16FS $|$ DB18FS $|$ DB20FS & SILence & \\ \hline
\hspace{1em}:TIMing 	& PAL $|$ NTSC1 $|$ NTSC2 $|$ NTSC3 $|$ NTSC4 $|$ NTSC5 & PAL & \\ \hline
\hspace{1em}:VERSion? & - & - & $<$query only$>$ \\ \hline
:AUDio1? 							& - & - & $<$query only$>$ \\ \hline


& & & \\ \hline
:AUDio2								&	&	&	\\ \hline
\hspace{1em}:SIGnal 	& S800Hz $|$ S1kHz $|$ SEBu1kHz $|$ SBBc1kHz $|$ MEBU1kHz $|$ M1kHz $|$ DUAL $|$ F48kHz $|$ WORDclock & S800Hz & \\ \hline
\hspace{1em}:LEVel 		& SILence $|$ DB0FS $|$ DB9FS $|$ DB12FS $|$ DB15FS $|$ DB16FS $|$ DB18FS $|$ DB20FS & SILence & \\ \hline
\hspace{1em}:TIMing 	& PAL $|$ NTSC1 $|$ NTSC2 $|$ NTSC3 $|$ NTSC4 $|$ NTSC5 & PAL & \\ \hline
\hspace{1em}:VERSion? & - & - & $<$query only$>$ \\ \hline
:AUDio2? 							& - & - & $<$query only$>$ \\ \hline


& & & \\ \hline
:TIMeclock						& & & \\ \hline
\hspace{1em}:DFORmat 	& DMY $|$ MDY $|$ YMD & DMY & \\ \hline
\hspace{1em}:DATe			& $<$year$>$,$<$month$>$,$<$date$>$ & 99,5,1 & \\ \hline
\hspace{1em}:TFORmat	& HOUR12 $|$ HOUR24 & HOUR24 & \\ \hline
\hspace{1em}:TIMe			& $<$hour$>$,$<$Minute$>$,$<$second$>$ & 8,0,0 & \\ \hline
\hspace{1em}:REFerence	& LTC $|$ VITC $|$ VFFRrequency $|$ REF1HZ $|$ INTernal & LTC & \\ \hline
\hspace{1em}:OFFSet		& $<$offset$>$ & 0 & \\ \hline
\hspace{1em}:VERSion?	& - & - & $<$query only$>$ \\ \hline
:TIMeclock?						& - & - & $<$query only$>$ \\ \hline


& & & \\ \hline
:TLG1-8								&	&	&	\\ \hline
\hspace{1em}:SYStem		& OFF $|$ HD1080P60 $|$ HD1080P5994 $|$ HD1080P50 $|$ HD1080I30 $|$ HD1080I2997 $|$ HD1080I25 $|$ HD1080P30 $|$ HD1080P2997 $|$ HD1080P25 $|$ HD1080P24 $|$ HD1080P2398 $|$ HD1080sF30 $|$ HD1080sF2997 $|$ HD1080sF25 $|$ HD1080sF24 $|$ HD1080sF2398 $|$ HD720P60 $|$ HD720P5994 $|$ HD720P50 $|$ HD720P30 $|$ HD720P2997 $|$ HD720P25 $|$ HD720P24 $|$ HD720P2398 & & \\ \hline
\hspace{1em}:DELay	&	$<$Field$>,<$Line$>,<$HTime$>$	&	&	\\ \hline
\hspace{1em}:VERSion?	& - & - & $<$query only$>$ \\ \hline
:HD1-8									&	&	&	\\ \hline
\hspace{1em}:PATTern	& BLACk $|$ SDICheck $|$ PLUGe $|$ LRAMp $|$ CLAPperbrd $|$ COLOrbar $|$ COMBInation $|$ WINdow $|$ CROSshatch $|$ WHITe	&	 &	\\ \hline
\hspace{2em}:MOD	& HH $|$ HS $|$ SS, A105 $|$ A100 $|$ A95 $|$ A90 $|$ A85 $|$ A80 $|$ A75 $|$ A70 $|$ A65 $|$ A60 $|$ A55 $|$ A50 $|$ A45 $|$ A40 $|$ A35 $|$ A30 $|$ A25 $|$ A20 $|$ A15 $|$ A10 $|$ A5 $|$ A0 $|$ AM5 & & Applies only for certain patterns \\ \hline
\hspace{1em}:SYStem		& OFF $|$ HD1080I30 $|$ HD1080I2997 $|$ HD1080I25 $|$ HD1080P30 $|$ HD1080P2997 $|$ HD1080P25 $|$ HD1080P24 $|$ HD1080P2398 $|$ HD720P60 $|$ HD720P5994 $|$ HD720P50 $|$ HD720P30 $|$ HD720P2997 $|$ HD720P25 $|$ HD720P24 $|$ HD720P2398 $|$ SD525 $|$ SD625 & & \\ \hline
\hspace{1em}:EMBaudio	&	&	& \\ \hline
\hspace{2em}:SIGnal	&	SILence $|$ SINE $|$ CLICK $|$ OFF & & \\ \hline
\hspace{2em}:LEVel	&	DB0FS $|$ DB6FS $|$ DB12FS $|$ DB18FS $|$ DB24FS & & \\ \hline
\hspace{2em}:CLIck	&	-499 to 500	&	& \\ \hline
\hspace{1em}:TEXT		&	&	&	\\ \hline
\hspace{2em}:STRing1	&	``TEXT1'' &	&	\\ \hline
\hspace{2em}:STRing2	&	``TEXT2'' &	&	\\ \hline
\hspace{2em}:STRing3	&	``TEXT3'' &	&	\\ \hline
\hspace{2em}:MOVement	&	OFF $|$ VERtical $|$ HORizontal $|$ BOTH	&	&	\\ \hline
\hspace{2em}:SCAle	& 1 to 4	&	&	\\ \hline
\hspace{2em}:COLor	& WHIte $|$ YELlow $|$ CYAn $|$ GREen $|$ MAGenta $|$ BLUe $|$ BLAck &	& \\ \hline
\hspace{2em}:BACKground & WHIte $|$ YELlow $|$ CYAn $|$ GREen $|$ MAGenta $|$ BLUe $|$ BLAck &	& \\ \hline
\hspace{1em}:DELay	&	$<$Field$>,<$Line$>,<$HTime$>$	&	&	\\ \hline
\hspace{1em}:VERSion?	& - & - & $<$query only$>$ \\ \hline
:VERSion? 	& - & - & $<$query only$>$ \\ \hline


& & & \\ \hline
:DL1-4									&	&	&	\\ \hline
\hspace{1em}:PATTern	& BLACk $|$ SDICheck $|$ PLUGe $|$ LRAMp $|$ CLAPperbrd $|$ COLOrbar $|$ COMBInation $|$ WINdow $|$ CROSshatch $|$ WHITe	&	 &	\\ \hline
\hspace{2em}:MOD	& HH $|$ HS $|$ SS, A105 $|$ A100 $|$ A95 $|$ A90 $|$ A85 $|$ A80 $|$ A75 $|$ A70 $|$ A65 $|$ A60 $|$ A55 $|$ A50 $|$ A45 $|$ A40 $|$ A35 $|$ A30 $|$ A25 $|$ A20 $|$ A15 $|$ A10 $|$ A5 $|$ A0 $|$ AM5 & & Applies only for certain patterns \\ \hline
\hspace{1em}:SYStem		& OFF $|$ HD1080I30 $|$ HD1080I2997 $|$ HD1080I25 $|$ HD1080P30 $|$ HD1080P2997 $|$ HD1080P25 $|$ HD1080P24 $|$ HD1080P2398 $|$ HD1080sF30 $|$ HD1080sF2997 $|$ HD1080sF25 $|$ HD1080sF24 $|$ HD1080sF2398 $|$ HD720P60 $|$ HD720P5994 $|$ HD720P50 $|$ HD720P30 $|$ HD720P2997 $|$ HD720P25 $|$ HD720P24 $|$ HD720P2398 $|$ SD525 $|$ SD625 & & \\ \hline
\hspace{2em}:INTERFace	&	I1 $|$ I2 $|$ I3 $|$ I4 $|$ I5 $|$ I6 &	&	\\ \hline
\hspace{1em}:EMBaudio	&	&	& \\ \hline
\hspace{2em}:SIGnal	&	SILence $|$ SINE $|$ CLICK $|$ OFF & & \\ \hline
\hspace{2em}:LEVel	&	DB0FS $|$ DB6FS $|$ DB12FS $|$ DB18FS $|$ DB24FS & & \\ \hline
\hspace{2em}:CLIck	&	-499 to 500	&	& \\ \hline
\hspace{1em}:TEXT		&	&	&	\\ \hline
\hspace{2em}:STRing1	&	``TEXT1'' &	&	\\ \hline
\hspace{2em}:STRing2	&	``TEXT2'' &	&	\\ \hline
\hspace{2em}:STRing3	&	``TEXT3'' &	&	\\ \hline
\hspace{2em}:MOVement	&	OFF $|$ VERtical $|$ HORizontal $|$ BOTH	&	&	\\ \hline
\hspace{2em}:SCAle	& 1 to 4	&	&	\\ \hline
\hspace{2em}:COLor	& WHIte $|$ YELlow $|$ CYAn $|$ GREen $|$ MAGenta $|$ BLUe $|$ BLAck &	& \\ \hline
\hspace{2em}:BACKground & WHIte $|$ YELlow $|$ CYAn $|$ GREen $|$ MAGenta $|$ BLUe $|$ BLAck &	& \\ \hline
\hspace{1em}:DELay	&	$<$Field$>,<$Line$>,<$HTime$>$	&	&	\\ \hline
\hspace{1em}:VERSion?	& - & - & $<$query only$>$ \\ \hline


& & & \\ \hline
:LTCGenerator1-2	&	& & \\ \hline
\hspace{1em}:FORMat	&	$<$Format$>$,$<$Syncmode$>$,$<$Hour$>$,$<$Min$>$ & & 
							24FPS $|$ 25FPS $|$ 2997NOND $|$ 2997DROP $|$ 30FPS, NONE $|$ CONF $|$ AUTO, 0..23, 0..59  \\ \hline
\hspace{1em}:OFFSET & -5000000..4999999 & & \\ \hline
\hspace{1em}:TIMEZone	& $<$Hour$>$, $<$Min$>$ & & -11..+11, 0 $|$ 30 \\ \hline
\hspace{1em}:DAYLight & & & \\ \hline
\hspace{2em}:MODE & $<$Mode$>$, $<$State$>$ & & AUTO $|$ CONF $|$ AUTO, ON $|$ OFF \\ \hline
\hspace{2em}:START & $<$Month$>$, $<$Day$>$, $<$Hour$>$ & & 1..12, 1..31, 0..23 \\ \hline
\hspace{2em}:END & $<$Month$>$, $<$Day$>$, $<$Hour$>$ & & 1..12, 1..31, 0..23 \\ \hline


\end{longtable}


\end{landscape}

\subsection{Commands Explanation}

\subsubsection{Mandated Commands}

\textbf{*CLS CLEAR STATUS}\\
Clear the error queue. Reset of the event registers has NOT been implemented in this version.

\textbf{*ESE STANDARD EVENT STATUS ENABLE COMMAND}\\
The device accepts this command but does not respond to it.

\textbf{*ESE? STANDARD EVENT STATUS ENABLE QUERY}\\
The device accepts this command but does not respond to it.

\textbf{*ESR? STANDARD EVENT STATUS REGISTER QUERY}\\
The device accepts this command but does not respond to it.

\textbf{*IDN? IDENTIFICATION QUERY}\\
The response contains four fields:
\begin{itemize}
\item Field 1: Company name
\item Field 2: Product name
\item Field 3: Serial number (KUxxxxxxx)
\item Field 4: Firmware level, i.e. software revisions for Mainboard-OSC
\end{itemize}

Example:
\textit{*IDN? response: PTV,PT5230,KU123456,1.0-1.2}

\textbf{*OPC OPERATION COMPLETE}\\
The device accepts this command but does not respond to it.

\textbf{*OPC? OPERATION COMPLETE QUERY}\\
The device accepts this command but does not respond to it.

\textbf{*RST RESET}\\
Resets the device to factory preset status. The six presets are NOT reset, i.e. any user preset will NOT be erased. The internal errorqueue and the SCPI errorqueue will also be reset. Finally the device and any optional units will be reset.

\textbf{*SRE SERVICE REQUEST ENABLE}\\
The device accepts this command but does not respond to it.

\textbf{*SRE? SERVICE REQUEST ENABLE QUERY}\\
The device accepts this command but does not respond to it.

\textbf{*STB? READ STATUS BYTE QUERY}\\
The device accepts this command but does not respond to it.

\textbf{*TST? SELF-TEST QUERY}\\
The device accepts this command but does not respond to it.

\textbf{*WAI WAIT TO CONTINUE}\\
The device accepts this command but does not respond to it.

\subsubsection{Required Commands}

\textbf{SYSTem commands}

\textbf{SYSTem:ERRor?}\\
Command for reading an SCPI error message from the error queue. See Chapter \ref{cha:error_codes} for a complete list of error codes.

Example:\\
\textit{SYST:ERR? response: -102,``Syntax error''}

\textbf{SYSTem:VERSion?}\\
Command for reading the SCPI version to which the RS232 implementation complies.

Example:\\
\textit{SYST:VERS? response: 1995.0}

\textbf{SYSTem:PRESet[:RECall]}\\
Command to recall a stored generator configuration from a preset. Six user presets from 1 to 6 are available.

Example:\\
\textit{SYST:PRES:REC 3}\\
recall preset 3 in the generator

\textit{SYST:PRES:REC?}\\
response: 3, i.e. preset 3 is currently active

\textbf{SYSTem:PRESet:STORe}\\
Command to store the actual configuration in a preset. Six user presets from 1 to 6 are available.

Example:\\
\textit{SYST:PRES:STOR 6}\\
store configuration in preset 6

\textbf{SYSTem:PRESet:NAMe}\\
Command for naming a user preset. Six user presets from 1 to 6 are available. Number of characters in the name are limited to sixteen, 16.

Example:\\
\textit{SYST:PRES:NAME 2,``WHAT'''}\\
name preset number 2 ``WHAT'

\textit{SYST:PRES:NAME? 2}\\
response: ``WHAT''

\textbf{SYSTem:PRESet:DOWNload}\\
Command for downloading, i.e. reading a complete preset from a PT5300 . Six user presets from 1 to 6 are available.\NoTelnetSupport

Example:\\
\textit{SYST:PRES:DOWN 4}\\
download content of preset 4

\textbf{SYSTem:PRESet:UPLoad}\\
Command for downloading, i.e. reading a complete preset from a PT5300 . Six user presets from 1 to 6 are available.\NoTelnetSupport

Example:\\
\textit{SYST:PRES:UP 4, \#aaa\ldots}\\
upload block data aaa to preset 4

\textbf{SYSTem:DOWNload}\\
Command for downloading, i.e. reading a complete PT5300 configuration incl. all presets.\NoTelnetSupport

Example:\\
\textit{SYST:DOWN}\\
download the complete PT5300

\textbf{SYSTem:UPLoad}\\
Command for uploading, i.e. storing a complete PT5300 configuration incl. all presets.\NoTelnetSupport
Example:\\
\textit{SYST:UP \#aaa\ldots}\\
upload block data aaa to PT5300

\textbf{STATus commands}

\textbf{STATus:OPERaction[:EVENT]?}\\
The device accepts this command but does not respond to it.

\textbf{STATus:OPERation:CONDition?}\\
The device accepts this command but does not respond to it.

\textbf{STATus:OPERation:ENABle}\\
The device accepts this command but does not respond to it.

\textbf{STATus:QUEStionable[:EVENt]?}\\
The device accepts this command but does not respond to it.

\textbf{STATus: QUEStionable:CONDition?}\\
The device accepts this command but does not respond to it.

\textbf{STATus: QUEStionable:ENABle}\\
The device accepts this command but does not respond to it.

\textbf{STATus:PT5300?}\\
Command to read the internal error status of the generator. If errors are detected use the command:

\textbf{DIAGnostic:ERRorqueue?}\\
to read the specific error. 

Response Description:

\begin{tabular}{|p{10em}|p{22em}|}
\hline
``No errors'' 			& No errors have occurred after power up. \\ \hline
``Active error''		& The generator presently has an error. \\ \hline
``No active error''	& The generator presently has no error, but one or more errors have been detected after power up. \\
\hline
\end{tabular}

Example:\\
\textit{STAT:PT5300?}\\
response: ``No active error''

\subsubsection{Instrument Commands}

\textbf{DIAGnostic commands}

\textbf{DIAGnostic:DISPlay}\\
The device accepts this command but does not respond to it.

\textbf{DIAGnostic:ERRorqueue:RESet}\\
Command to reset the internal error queue of the generator. The errorqueue is a circular queue consisting of five entries.

Example:\\
\textit{DIAG:ERR:RES}\\
reset the five elements in the errorqueue

\textbf{DIAGnostic:ERRorqueue?}\\
Command to read an entry in the errorqueue and point to next entry in the errorqueue. This command should be executed five times to read the complete errorqueue.

Example:\\
\textit{DIAG:ERR?}\\
response: -108,``Parameter not allowed''

\textbf{DISPlay commands}

\textbf{DISPlay:CONTrast}\\
The device accepts this command but does not respond to it.

\textbf{INPut commands}

\textbf{INPut:GENLock:INPut}\\
Command for selecting the genlock input. Possible selections are 

\begin{tabular}{|l|l|}
\hline
\textbf{Input:} & \textbf{Description:} \\ \hline
A & A \\ \hline
B & B \\ \hline
A\_B & A-B, i.e. loop through \\ \hline
SDI & SDI Genlock, (ONLY available with option PT 8606)\\ \hline
INTernal & Internal\\ 
\hline
\end{tabular}

When selecting a new input, the system for that particular input will apply.

Example:\\
\textit{INP:GENL:INP A\_B}\\
select input A/B as the genlock signal\\
\textit{INP:GENL:INP?}\\
response: A\_B

\textbf{INPut:GENLock:SYSTem}\\
Command for selecting the genlock system. Possible selections are 

%\begin{center}
\begin{tabular}{|l|l|l|l|}
\hline
\textbf{System:} 	& \textbf{A, B\& A\_B} & \textbf{SDI:} & \textbf{Description:} \\
PALBurst					& X 									&								&	PAL burst lock 		\\ \hline
NTSCburst					& X 									&								& NTSC burst lock		\\ \hline
SYNC625						& X										& 							& 625 sync lock			\\ \hline
SYNC525						& X										&								& 525 sync lock			\\ \hline
SDI625						&											& X							& 625/50 lock				\\ \hline
SDI525						&											& X							& 525/59.95 lock		\\ \hline
F358MHz						& X										&								& 3.58 MHz lock			\\ \hline
F443MHz						& X										& 							& 4.43 MHz lock			\\ \hline
F5MHz							& X										&								& 5 MHz lock				\\ \hline
F10MHz						& X										&								& 10 MHz lock				\\ \hline
\end{tabular}
%\end{center}

\textbf{Note:} When the input has been selected as Internal or Internal2, issuing this command will result in an error, namely: -200, ``Execution error''. This error will also occur if selecting a system which is invalid for the active input.

Example:\\
\textit{INP:GENL:SYST F358MHz}\\
set system to 3.59 MHz clock\\
\textit{INP:GENL:SYST?}\\
response: F358 MHz

\textbf{INPut:GENLock:DELay}
Command to set the delay for the genlock input. The delay is defined by three parameters $<$Field$>$,$<$Line$>$,$<$HTime$>$
where $<$Field$>$ sets the field offset, $<$Line$>$ sets the line offset and $<$HTime$>$ sets the horizontal time in ns, i.e.

\begin{itemize}
\item HTime(PAL$<$64000.0ns 
\item HTime(NTSC)$<$63492.1ns
\end{itemize}

\textbf{Note:} It is not possible to select timing when the genlock system is 3.58 MHz, 4,43 MHz, 5 MHz, or 10 MHz or the input is set to internal or internal2. This will result in an execution error, namely: -200,``Execution error''. 

Also it is not possible to select a delay outside the range of the selected system. See table below:

%\begin{tabular}{|p{4em}|p{4em}|p{4em}|p{4em}|p{4em}|p{4em}|p{4em}|p{4em}|}
\begin{tabular}{|l|l|l|l|l|l|l|l|}
\hline
\multicolumn{4}{|c|}{Analogue} & \multicolumn{4}{|c|}{Digital} \\ 
\hline
\multicolumn{2}{|c|}{PAL, 625 Lines} & \multicolumn{2}{|c|}{NTSC, 625 Lines} & \multicolumn{2}{|c|}{D1, 625 Lines} & \multicolumn{2}{|c|}{D1, 525 Lines} \\ 
\hline
Field: 	& Line: 			& Field: 				& Line: 			& Field:	& Line:			 	& Filed: 	& Line: \\ \hline
-3 			& -0,..,-312	& -							& -						& -				& -						& -				& - \\ \hline
-2			& -0,..,-311	& -							& -						& -				& -						& -				& - \\ \hline
-1			& -0,..,-312	& -1 						& -0,..,-262	& -				& -						& -				& -  \\ \hline
-0			& -0,..,+311	& -0 						& -0,..,-261	& -0			& -0,..,-312	& -0			& -0,..,-262 \\ \hline
+0			& +0,..,+312	& +0						& -0,..,+262	& +0			& +0,..,+311	& +0			&	-0,..,+261 \\ \hline
+1			& +0..,+311		& +1						& -0,..,+261	& +1			& +0					& +1			& +0  \\ \hline
+2			& +0..,+312		& +2						& +0					& -				& -						& -				& - \\ \hline
+3			& +0..,+311		& -							& -						& -				& -						& -				& - \\ \hline
+4			& +0					& -							& -						& -				& -						& -				& - \\ \hline
\end{tabular}

Example:\\
\textit{INP:GENL:DEL+2,+5,+123.5}\\
set delay to 2 field, 5 line \& 123,5 ns\\
\textit{INP:GENL:DEL?}\\
response: +2,+005,+00123.5

\textbf{INPut:GENLock?}\\
Command to display the status and the settings of the genlock. The respond is defined as:
$<$lock info$>$,$<$input$>$,$<$system$>$,$<$Field$>$,$<$Line$>$,$<$HTime$>$ where $<$lock info$>$ is either GENLOCKED or UNLOCKED. 

For an explanation concerning the rest of the response see the commands: 

INP:GENL:INP, INP:GENL:SYST and INP:GENL:DEL.

\textbf{Note:} The response will always return the above six parameters. But when selecting the input as INTERNAL the parameters $<$system$>$,$<$Field$>$,$<$Line$>$,$<$HTime$>$ will have no meaning. Also when selecting the system as a timing, e.g. 3.58 MHz, the parameters $<$Field$>$,$<$Line$>$,$<$HTime$>$ will have no meaning. In these cases the returned values should be discarded and only the relevant parameters should be used.

Example:\\
\textit{INP:GENL?}\\
response: UNLOCKED,A,NTSCBURST,+1,212,00000.2\\
\textit{INP:GENL?}\\
response: UNLOCKED,A,F358 MHz,+0,+0,+0\\
\textit{INP:GENL?}\\
response: UNLOCKED,INTERNAL,NA, +0,+0,+0

\textbf{INPut:SDIGenlock:VERSion?}\\
Command to display the version of the optional PT 8606 SDI Genlock. The response contains four fields:
\begin{itemize}
\item Field 1: Company name
\item Field 2: Type name
\item Field 3: Serial number (KU number)
\item Field 4: Not available for this option, i.e. the returned value is 0.
\end{itemize}

Example:\\
\textit{INP:SDIG:VERS?}\\
Response: PTV,PT 8606,KU123456,0

\textbf{OUTPut commands}

\textbf{OUTPut:BB1:SYSTem}\\
\textbf{OUTPut:BB2:SYSTem}\\
Command to select the system of the standard Black Burst module. Systems available are:

\begin{tabular}{|l|l|}
\hline
\textbf{System:} & \textbf{Description:} \\ \hline
PAL & PAL \\ \hline
PAL\_ID & PAL with line 7 pulse \\ \hline
NTSC & NTSC with setup \\ 
\hline
\end{tabular}

Example:\\
\textit{OUTP:BB1:SYSTPAL\_ID}\\
set system for BB module 1 to PAL with line 7 pulse\\
\textit{OUTP:BB1:SYST?}\\ 
response: PAL\_ID

\textbf{OUTPut:BB1:DELay}\\
\textbf{OUTPut:BB2:DELay}\\
Command to set the delay of the standard Black Burst module. The delay is defined by three parameters: $<$Field$>$,$<$Line$>$,$<$HTime$>$
where $<$Field$>$ sets the field offset, $<$Line$>$ sets the line offset and $<$HTime$>$ sets the horizontal time in ns, i.e.
\begin{itemize}
\item HTime(PAL)$<$64000.0ns
\item HTime(NTSC)$<$63492.1ns
\end{itemize}

\textbf{Note:} It is not possible to select a delay outside the range of the selected system. See table below:

\begin{tabular}{|l|l|l|l|}
\hline
\multicolumn{4}{|c|}{Analogue} \\ 
\hline
\multicolumn{2}{|c|}{PAL, 625 Lines} & \multicolumn{2}{|c|}{NTSC, 625 Lines} \\ 
\hline
Field: 	& Line: 			& Field: 				& Line: 			\\ \hline
-3 			& -0,..,-312	& -							& -						\\ \hline
-2			& -0,..,-311	& -							& -						\\ \hline
-1			& -0,..,-312	& -1 						& -0,..,-262	\\ \hline
-0			& -0,..,+311	& -0 						& -0,..,-261	\\ \hline
+0			& +0,..,+312	& +0						& -0,..,+262	\\ \hline
+1			& +0..,+311		& +1						& -0,..,+261	\\ \hline
+2			& +0..,+312		& +2						& +0					\\ \hline
+3			& +0..,+311		& -							& -						\\ \hline
+4			& +0					& -							& -						\\ \hline
\end{tabular}

Example:\\
\textit{OUTP:BB2:DEL-0,-0,-3245.2}\\
set delay for BB module 2 to -2 field, -4 line \& -3245.2ns\\
\textit{OUTP:BB2:DEL?}\\
response: -2,-004,-03245.2\\

\textbf{OUTPut:BB1:SCHPhase}\\
\textbf{OUTPut:BB2:SCHPhase}\\
Command to set the ScH-Phase of the standard Black Burst module. The ScH-Phase value must be in the range: -179$<$ScH-Phase$<$=+180

Example:\\
\textit{OUTP:BB2:SCHP-160}\\
set the ScHPhase for BB module 2 to -160deg\\
\textit{OUTP:BB2:SCHP?}\\
response: -160

\textbf{OUTPut:BB1:VERSion?}\\
\textbf{OUTPut:BB2:VERSion?}\\
Command to display the version of the standard Black Burst module. The response contains four fields:
\begin{itemize}
\item Field 1: Company name
\item Field 2: Type name, which in this case is NA, not available
\item Field 3: Serial number (KUxxxxxx)
\item Field 4: Software version for the Black Burst module
\end{itemize}

\textbf{Note:} The response from this command is identical for both BB module 1 and 2. 

Example:\\
\textit{OUTP:BB1:VERS?}\\
response: PTV,NA,KU123456,2.1

\textbf{OUTPut:BB1?}\\
\textbf{OUTPut:BB2?}\\
Command to display the complete settings of the standard Black Burst modules. The response contains five fields: $<$System$>$,$<$Field$>$,$<$Line$>$,$<$HTime$>$,$<$ScHPhase$>$

For an explanation of the response, see the commands: 

OUTP:BBn:SYST,OUTP:BBn:DEL and OUTP:BBn:SCHP, where n:1 or 2

Example:\\
\textit{OUTP:BB1?}\\
response: PAL,+2+123,+12345.5,-160

\textbf{OUTPUT:ATPGenerator2:PATTern}\\
\textbf{OUTPUT:ATPGenerator5:PATTern}\\
Command to select the pattern of an optional PT 8631 Analog Test Pattern Generator. Patterns available are:

\begin{tabular}{|l|l|l|l|}
\hline
\textbf{Pattern:}	& \textbf{PAL:}	& \textbf{NTSC:}	& \textbf{Description:} \\ \hline
CBSMpte		& 			&	X			& SMPTE Colour Bar \\ \hline
CBEBu			& X			& 			& EBU Colour Bar \\ \hline
CBFCc			&				& X			& FCC Colour Bar \\ \hline
CB100			& X			& 			& 100\% Colour Bar \\ \hline
CBGRey75	& X			&				& Split field Colour bar w/75\% grey \\ \hline
CBRed75		& X			&				& Split field Colourbar w/75\% red \\ \hline
RED75			& X			& X			& 75\% Red \\ \hline
LSWeep		& X			& X			& Luminance sweep \\ \hline
MPULse		& X			& X			& Multipulse \\ \hline
SINXx			& X			& X			& Sinx/x \\ \hline
CCIR18		& X			&				& CCIR line 18 \\ \hline
NCMB			& 			& X			& NTC7 Combination \\ \hline
FCCMburst	& 			& X			& FCC Multiburst \\ \hline
WIN15			& X 		& X			& Window 15\% \\ \hline
WIN20			& X			& X			& Window 20\% \\ \hline
WIN100		& X			& X			& Window 100\% \\ \hline
GREy50		& X			& X			& Grey 50\% \\ \hline
WHITe100	& X			& X			& White 100\% \\ \hline
BLACkburst& X			& X			& Black Burst \\ \hline
FSWave		& X			& X			& Field square wave \\ \hline
BLWH01		& X			& X			& 0.1Hz Black/white \\ \hline
RAMP			& X			& X			& Ramp \\ \hline
MRAMp			& X			& X 		& Ramp Modulated \\ \hline
STAircase5& X			& X			& Staircase 5 step \\ \hline
MSTaircase5& X		& X			& Staircase 5 step modulated \\ \hline
STAircase10& X		& X			& Staircase 10 step \\ \hline
PBAR			& X			& X			& Pulse \& Bar \\ \hline
CCIR17		& X 		& 			& CCIR line 17\\ \hline
CCIR330		& X			&				& CCIR line 330\\ \hline
CCIR331		& X			&				& CCIR line 331\\ \hline
FCCComposite	&		& X			& FCC Composite\\ \hline
NCMP			&				& X			& NTC7 Composite\\ \hline
PHILips43	& X			&				& Philips pattern 4:3 format\\ \hline
PHILips169	& X		&				& Philips pattern 16:9 format\\ \hline
FUBK43		&	X			&				& FuBK pattern 4:3 format\\ \hline
FUBK169		& X			&				& FuBK pattern 16:9 format\\ \hline
CROSshatch	& X		& X			& Cross Hatch\\ \hline
CROSshatch169	& X	& X			& Cross Hatch in 16:9\\ \hline
CIRCl43		& X			&				& White circle on black in 4:3\\ \hline
CIRCl169	& X			& 			&	White circle on black in 16:9\\ \hline
PLUGe			& X			& X			& Pluge\\ \hline
SAFerea		& X			& X			& Safe area\\ \hline
SWAVe250	& X			& X			& Squarewave 250kHz\\ \hline
VMT01			& X			&				& VMT01 testpattern\\ \hline
\end{tabular}

\textbf{Note:} Not all the patterns are available in both systems. Trying to select a pattern not available in the active system will result in an error, namely: -200, ``Execution error''.

Example:\\
\textit{OUTP:ATPG2:PATT PHIL43}\\
set the pattern in the ATPG module to PHILIPS 4:3\\
\textit{OUTP:ATPG2:PATT?}\\
response: PHILIPS43

\textbf{OUTPUT:ATPGenerator2:PATTern:MODify:APAL}\\
\textbf{OUTPUT:ATPGenerator2:PATTern:MODify:PLUGE}\\
\textbf{OUTPUT:ATPGenerator2:PATTern:MODify:STAircase10}

\textbf{OUTPUT:ATPGenerator5:PATTern:MODify:APAL}\\
\textbf{OUTPUT:ATPGenerator5:PATTern:MODify:PLUGE}\\
\textbf{OUTPUT:ATPGenerator5:PATTern:MODify:STAircase10}\\
Commands to enable/disable a modification of a/the complex pattern in an optional PM8631 Analog test pattern Generator. The possible selections are OFF and ON.

\textbf{Note:} The above modification are only available when the Philips 4:3 pattern has been selected. Trying to modify any other pattern will result in an error, namely. -200, ``Execution error''.

Example:\\
\textit{OUTP:ATPG2:PATT:MOD:APAL OFF}\\
remove anti-PAL from Philips pattern in the ATPG2\\
\textit{OUTP:ATPG2:MOD:APAL?}\\
response: OFF

\textbf{OUTPut:ATPGenerator2:TEXT:STRing1}\\
\textbf{OUTPut:ATPGenerator2:TEXT:STRing2}\\
\textbf{OUTPut:ATPGenerator2:TEXT:STRing3}

\textbf{OUTPut:ATPGenerator2:TEXT:STRing1}\\
\textbf{OUTPut:ATPGenerator2:TEXT:STRing2}\\
\textbf{OUTPut:ATPGenerator2:TEXT:STRing3}\\
Command to insert one or more text strings into the pattern of the optional PT8631 Analog Test pattern Generator. Three parameters are possible, i.e. OFF, ON and some text, ``TEXT''. The string being edited depends upon the pattern selected. One group of patterns are the
standard patterns, e.g. 75\% Red, Colourbar etc. and another group is the complex pattern which is the Philips 4:3 pattern. The standard patterns will have three lines of text available, while the complex pattern only have two lines of text.

\textbf{Note:} To switch the text on/off use the parameters: ON or OFF. To alter the actual text: use ``TEXT''. The text is limited to sixteen characters. % For a list of available characters, please refer to Appendix xxx.

Example:\\
\textit{OUTP:ATPG2:TEXT:STR1 ``ANALOG''}\\
set text line 1 in ATPG2 to ANALOG

\textit{OUTP:ATPG2:TEXT:STR1 ON}\\
switch text in the pattern ON

\textit{OUTP: ATPG2:TEXT:TEXT?}\\
response: ON,``ANALOG''

\textbf{OUTPut:ATPGenerator2:TEXT:STYLe}\\
\textbf{OUTPut:ATPGenerator5:TEXT:STYLe}\\
Command to select how the text is to be inserted into the Philips 4:3 pattern in the optional PT8631 Analog test Pattern generator. The possible selections are STANdard or COMPlex. When choosing the standard style, the two text lines will be placed in the lower right corner.
When choosing the complex style, the text will be placed in the upper and lower text fields in the Philips pattern.

\textbf{Note:} This command is only available with the Philips 4:3 pattern. Attempting to use the command for any other pattern will result in an error, namely. -200, ``Execution error''.

Example:\\
\textit{OUTP:ATPG2:TEXT:STYL COMP}\\
set text style in ATPG2 to complex

\textit{OUTP: ATPG2:TEXT:STYL?}\\
response: COMPLEX

\textbf{OUTPut:ATPGenerator2:TEXT:CLOCk}\\
\textbf{OUTPut:ATPGenerator5:TEXT: CLOCk}\\
Command to insert time/date information into a pattern in the optional PT8631 Analog test Pattern The possible selections are:

\begin{tabular}{|l|l|}
\hline
 & Description \\
\hline
OFF 	& No time- or date-information \\ \hline
TIMe	& Time information \\ \hline
DTIMe	& Time- and date-information \\ 
\hline
\end{tabular}

\textbf{Note:} This command requires the optional PT8637 Time Clock Interface to be present.

Example:\\
\textit{OUTP:ATPG5:TEXT:CLOC TIM}\\
insert time int pattern in ATPG module5

\textit{OUTP: ATPG5:TEXT:CLOC?}\\
response: TIME

\textbf{OUTPut:ATPG2:SYSTem}\\
\textbf{OUTPut:ATPG5:SYSTem}\\
Command to select the system of an optional PT 8631 Analog Test Pattern Generator. Systems available are:

\begin{tabular}{|l|l|}
\hline
\textbf{System:} & \textbf{Description:} \\ \hline
PAL & PAL \\ \hline
PAL\_ID & PAL with line 7 pulse \\ \hline
NTSC & NTSC with setup \\ \hline
\end{tabular}

\textbf{Note:} If the pattern becomes invalid when selecting a new system, the pattern will change according to:

\begin{tabular}{|l @{ $\rightarrow$ } l|}
\hline
\multicolumn{2}{|c|}{PAL specific patterns:} \\ \hline
EBU C.Bar & SMPTE C.Bar\\ \hline
100\% C.Bar & SMPTE C.Bar\\ \hline
75\% C.Bar+ Grey & SMPTE C.Bar\\ \hline
75\% C.Bar+ Red & SMPTE C.Bar\\ \hline
CCIR Line 18 & FCC Multiburst\\ \hline
CCIR Line 17 & FCC Composite\\ \hline
CCIR line 330 & FCC Composite\\ \hline
CCIR Line 331 & FCC Composite\\ \hline
Philips 4:3 & Crosshatch 4:3\\ \hline
VMT01 & Crosshatch 4:3\\ \hline
\multicolumn{2}{|c|}{NTSC specific patterns:} \\ \hline
SMPTE C.Bar & EBU C.Bar\\ \hline
FCC C.Bar & EBU C.Bar\\ \hline
NTC Combination & CCIR Line 18\\ \hline
FCC Multiburst & CCIR Line 18\\ \hline
FCC Composite & CCIR Line 17\\ \hline
NTC7 Composite & CCIR Line 17\\ \hline
\end{tabular}

Example:\\
\textit{OUTP:ATPG2:SYSTE PAL\_ID}\\
set the system in the generator to PAL with line 7 pulse

\textit{OUTP:ATPG2:SYST?}\\
response: PAL\_ID

\textbf{OUTPut:ATPGenerator2:DELay}\\
\textbf{OUTPut:ATPGenerator5:DELay}\\
Command to set the delay of an optional PT 8631 Analog Test Pattern Generator. The delay is defined by five parameters: $<$Field$>$,$<$Line$>$,$<$HTime$>$, where $<$Field$>$ sets the field offset, $<$Line$>$ sets the line offset and $<$HTime$>$ sets the horizontal time in ns, i.e.

\begin{itemize}
\item HTime(PAL) <64000.0ns
\item HTime(NTSC)<63492.1ns
\end{itemize}

\textbf{Note:} It is not possible to select a delay outside the range of the selected system. See table below:

\begin{tabular}{|p{4em}|p{4em}|p{4em}|p{4em}|}
\hline
\multicolumn{4}{|c|}{Analogue} \\ 
\hline
\multicolumn{2}{|c|}{PAL, 625 Lines} & \multicolumn{2}{|c|}{NTSC, 625 Lines} \\ 
\hline
Field: 	& Line: 			& Field: 				& Line: 			\\ \hline
-3 			& -0,..,-312	& -							& -						\\ \hline
-2			& -0,..,-311	& -							& -						\\ \hline
-1			& -0,..,-312	& -1 						& -0,..,-262	\\ \hline
-0			& -0,..,+311	& -0 						& -0,..,-261	\\ \hline
+0			& +0,..,+312	& +0						& -0,..,+262	\\ \hline
+1			& +0..,+311		& +1						& -0,..,+261	\\ \hline
+2			& +0..,+312		& +2						& +0					\\ \hline
+3			& +0..,+311		& -							& -						\\ \hline
+4			& +0					& -							& -						\\ \hline
\end{tabular}

Example:\\
\textit{OUTP:ATPG2:DeL -2,-4,-3245.2}\\
set the delay in the generator to -2 field, -4 line \& -3245.2ns

\textit{OUTP:ATPG2:DEL?}\\
response:-2,-004,-03245.2\\

\textbf{OUTPut:ATPGenerator2:SCHPhase}\\
\textbf{OUTPut:ATPGenerator5:SCHPhase}\\
Command to set the ScH-Phase of an optional PT 8631 Analog Test Signal Generator. The ScH-Phase value must be in the range: -180$<$ScH-Phase$<$=+180

Example:\\
\textit{OUTP:ATPG2:SCHP -123}\\
set the ScH-Phase in the generator to -123deg\\
\textit{OUTP:ATPG2:SCHP?}\\
response: -123

\textbf{OUTPut:ATPGenerator2:VERSion?}\\
\textbf{OUTPut:ATPGenerator5:VERSion?}\\
Command to display the version of an optional PT 8631 Analog Test Pattern Generator. The response contains four fields:
\begin{itemize}
\item Field 1: Company name
\item Field 2: Type name
\item Field 3: KU number
\item Field 4: Software version for the analog test pattern generator
\end{itemize}

Example:\\
\textit{OUTP:ATPG2:VERS?}\\
response: PTV,PT8631,KU093456,1.0

\textbf{OUTPut:ATPGenerator?}\\
Command to display the complete settings of an optional PT 8631 Analog Test Pattern Generator. The response contains eight fields: 
$<$Pattern$>$, $<$Text insertion$>$, $<$System$>$, $<$Field$>$, $<$Line$>$, $<$HTime$>$, $<$ScHPhase$>$

For an explanation of the response, see the commands: 

OUTP:ATPG2n:PATT, OUTP:ATPG2n:TEXT,OUTP:ATPG2n:SYST, OUTP:ATPGn:DEL, and OUTP:ATPGn:SCHP, where n: 2 or 5

\textbf{Note:} The field text insertion simply gives the information whether there is text or clock in the pattern selected, the text itself is NOT returned. The information about the the pattern modifications is not returned.

Example:\\
\textit{OUTP:ATPG2?}\\
response: CBEBU,OFF,PAL,+2,+123,+12345

\textbf{OUTPut:STGenerator2:PATTern}\\
\textbf{OUTPut:STGenerator3:PATTern}\\
\textbf{OUTPut:STGenerator4:PATTern}\\
Command to select the pattern of an optional PT 8639 SDI Test Signal generator. Patterns available are:

\begin{tabular}{|l|l|l|l|}
\hline
\textbf{Pattern:}	& \textbf{SDI625:}	& \textbf{SDI525:}	& \textbf{Description:} \\ \hline
CBSMpte						& 	& X	& SMPTE Colour Bar\\ \hline
CBEBu							& X &		& EBU Colour Bar\\ \hline
CBFCc							& 	& X & FCC Colour Bar\\ \hline
CBEBu8						& X & X	& EBU Colour Bar, ITU801\\ \hline
CB100							& X &		& 100\% Colour Bar\\ \hline
CBRed75						& X & 	& 75\% C.Bar + Red\\ \hline
RED75							& X & X	& 75\% Red\\ \hline
MULTiburst				& X & X	& Multiburst\\ \hline
WIN15							& X & X	& Window 15\%\\ \hline
WIN20							& X & X & Window 20\%\\ \hline
WIN100						& X & X & Window 100\%\\ \hline
GREy50						& X & X & Grey 50\%\\ \hline
BLACk							& X & X & Black\\ \hline
SDICheck					& X & X & SDI Check Field\\ \hline
DGRey							& X & X & Digital Grey\\ \hline
STAircase5				& X & X & Staircase 5 step\\ \hline
CROSshatch				& X & X & Cross Hatch\\ \hline
PLUGe							& X & X & Pluge\\ \hline
\end{tabular}

\textbf{Note:} Not all the patterns are available in both systems. Trying to select a pattern not available in the active system will result in an error, namely: -200,``Execution error''

Example:\\
\textit{OUTP:STSG3:PATT CSBM}\\
set the pattern in STSG module 3 to SMPTE Colour Bar\\
\textit{OUTP: STSG3:PATT?}\\
response: CBSMPTE

\textbf{OUTPut:STSGenerator2:SYSTem}\\
\textbf{OUTPut:STSGenerator3:SYSTem}\\
\textbf{OUTPut:STSGenerator4:SYSTem}\\
Command to select the pattern of an optional PT 8639 SDI Test signal Generator in the PT5300. Systems available are:

\begin{tabular}{|l|l|}
\hline
\textbf{System:} & \textbf{Description:} \\ \hline
SDI625 & 625/50 system \\ \hline
SDI525 & 525/59.94 system \\ \hline
\end{tabular}

\textbf{Note:} If the pattern becomes invalid when selecting a new system, the pattern will change according to:

\begin{tabular}{|l@{$\rightarrow$}l|}
\hline
\multicolumn{2}{|l|}{625/50 specific patterns:} \\ \hline
EBU C.Bar & SMPTE C.Bar \\ \hline
75\% C.Bar+Grey & SMPTE C.Bar \\ \hline
\multicolumn{2}{|l|}{525/59.94 specific patterns:} \\ \hline
SMPTE C.Bar & EBU C.Bar \\ \hline
FCC C.Bar & EBU C.Bar \\ \hline
\end{tabular}

Example:\\
\textit{OUTP:STSG3:SYST SDI525}\\
set the pattern in STSG module 3 to 525/59.94\\
\textit{OUTP: STSG3:SYST?}\\
response: SDI525

\textbf{OUTPut:STSGenerator2:DELay}\\
\textbf{OUTPut:STSGenerator3:DELay}\\
\textbf{OUTPut:STSGenerator4:DELay}\\
Command to set the delay of an optional PT 8639 SDI Test Signal Generator. The delay is defined by three parameters:
$<$Field$>$,$<$Line$>$,$<$HTime$>$, where $<$Field$>$ sets the field offset, $<$Line$>$ sets the line offset and $<$HTime$>$ sets the horizontal time in ns, i.e.
\begin{itemize}
\item HTime(PAL)$<$64000.0ns
\item HTime(NTSC)$<$63492.1ns
\end{itemize}

\textbf{Note:} It is not possible to select a delay outside the range of the selected system. See table below:

\begin{tabular}{|p{5em}|p{5em}|p{5em}|p{5em}|}
\hline
\multicolumn{4}{|c|}{Digital} \\ \hline
\multicolumn{2}{|c|}{D1, 625 Lines} & \multicolumn{2}{|c|}{D1, 525 Lines} \\ \hline
Field:	& Line:				& Field:	& Line: \\ \hline
-0			& -0,..,-312	& -0			& -0,..,-262 \\ \hline
+0			& +0,..,+311	& +0			& -0,..,+261 \\ \hline
+1			& +0					& +1			& +0 \\ \hline
\end{tabular}

Example:\\
\textit{OUTP:STSG2:DEL+0,0312,+74.0}\\
set the delay in STSG module 2 to +0 filed, +312 line \& +74.0ns\\
\textit{OUTP:STSG2:DEL?}\\
response: +0,+3122,+00074.2

\textbf{OUTPut:STSG2:EDHinsert}\\
\textbf{OUTPut:STSG3:EDHinsert}\\
\textbf{OUTPut:STSG4:EDHinsert}\\
Command to insert EDH into the output of an optional PT 8639 SDI Test Signal generator. Possible selections are ON or OFF.

Example:\\
\textit{OUTP: STSG2:EDH OFF}\\
set the EDH insertion in STSG module 2 to OFF
\textit{OUTP: STSG2:EDH?}\\
response: OFF

\textbf{OUTPut: STSG2:EMBaudio:SIGNal}\\
\textbf{OUTPut: STSG3:EMBaudio:SIGNal}\\
\textbf{OUTPut: STSG4:EMBaudio:SIGNal}\\
Command to select the embedded audio signal in an optional PT 8639 SDI SDI Test Signal generator. Possible selections are:

\begin{tabular}{|l|l|}
\hline
\textbf{Signal:} & \textbf{Description:} \\ \hline
Off		& Off \\ \hline
S1KHZ & Stereo 1 kHz \\ \hline
\end{tabular}

Example:\\
\textit{OUTP: STSG2:EMB:SIGN S1KHZ}\\
set the embedded audio in STSG module 2 to stereo 1kHz\\
\textit{OUTP: STSG2:EMB:SIGN?}\\
response: S1KHZ

\textbf{OUTPut: STSG2:EMBaudio:LEVel}\\
\textbf{OUTPut: STSG3:EMBaudio:LEVel}\\
\textbf{OUTPut: STSG4:EMBaudio:LEVel}\\
Command to select the embedded audio level in an optional PT 8639 SDI SDI Test Signal generator. Possible selections are:

\begin{tabular}{|l|l|}
\hline
\textbf{Signal:} & \textbf{Description:} \\ \hline
SILence	& Silence \\ \hline
DB0FS		& 0 dB\\ \hline
DB9FS		& -9 dB\\ \hline
DB15FS	& -15 dB\\ \hline
DB18FS	& -18 dB\\ \hline
\end{tabular}

Example:\\
\textit{OUTP: STSG2:EMB:LEV DB0FS}\\
set the embedded audio level in STSG module 2 to 0 dB\\
\textit{OUTP: STSG2:EMB:LEV?}\\
response: DB0FS

\textbf{OUTPut:STSG2:VERSion?}\\
\textbf{OUTPut: STSG3:VERSion?}\\
\textbf{OUTPut: STSG4:VERSion?}\\
Command to display the version of an optional PT8639 SDI Test signal Generator. The response contains four fields:
\begin{itemize}
\item Field 1: Company name
\item Field 2: Type name
\item Field 3: KU number
\item Field 4: Software version for the PT 8639 SDI Test Signal Generator
\end{itemize}

Example:\\
\textit{<OUTP: STSG4:VERS?}\\
response: PTV,PT8639,KU123456,2.0

\textbf{OUTPut: STSG2?}\\
\textbf{OUTPut: STSG3?}\\
\textbf{OUTPut: STSG4?}\\
Command to display the complete settings of an optional PT 8639 SDI Test Signal Generator. The response contains eight fields:
$<$Pattern$>$, $<$System$>$, $<$Field$>$, $<$Line$>$, $<$HTime$>$, $<$EDH$>$, $<$Audio signal$>$, $<$Audio level$>$. 

For an explanation of the response, see the commands: 

OUTP:STSG n:PATT, OUTP:STSGn:SYST, OUTP:STSGn:DeL, OUTP:STSGn:EDH, OUTP:STSGn:EMB:SIGN and
OUTP:STSGn:EMB:LEV, where n: 2, 3, or 4.

Example:\\
\textit{OUTP:STSG4?}\\
response:CBEBU,SDI625,+0,+001,+12345.5,OFF,OFF,SIL

\textbf{OUTPut:STPGenerator1:PATTern}\\
\textbf{OUTPut:STPGenerator2:PATTern}\\
\textbf{OUTPut:STPGenerator3:PATTern}\\
Command to select the pattern of an optional PT8632 and PT8633 SDI Test Pattern Generators. Patterns available are:
\begin{center}

\begin{tabular}{|l|l|l|l|l|l|}
\hline
\multirow{2}{*}{Pattern:} & \multicolumn{2}{|c|}{PT8632} & \multicolumn{2}{|c|}{PT8633} & \multirow{2}{*}{Description:} \\ \cline{2-5}
 													& 525 line & 625 line					 & 525 line & 625 line & \\ \hline
%													| 525				|		625				| 525				| 625				|
CBSMpte										&						&		X					& 					&						&	SMPTE C.Bar \\ \hline
CBEBu											&						& X						&						& X					& EBU C.Bar\\ \hline
CBFCc											& X					&							& X					&						& FCC C.Bar\\ \hline
DBEBu8										& X					& X						& X					& X					& EBU C.Bar 8 bit\\ \hline
CB100											& X					&							&						&						& 100\% C.Bar\\ \hline
CBGRey75									& X					&							& 					&						& Split field C.Bar +75\% grey\\ \hline
CBRed75										&						& X						&						& X					& Split field C.Bar +75\% red\\ \hline
RED75											& X					& X						&	X					& X					& 75\% Red\\ \hline
MULTiburst								& X					& X						& X					& X					& Multiburst\\ \hline
LSWeep										& X					& X						& X					& X					& Luminance sweep\\ \hline
YCRCbsweep								&						&							& X					& X					& Y, Cr, Cb sweep\\ \hline
MPULse										& X					& X						& X					& X					& Multipulse\\ \hline
SINXx											&  					&  						&	X					&	X					& Sinx/x\\ \hline
WIN15											& X					& X						&						&						& Window 15\%\\ \hline
WIN20											& X					& X						&						&						& Window 20\%\\ \hline
WIN100										& X					& X						&						&						& Window 100\%\\ \hline
WHITe100									&  					&  						&	X					&	X					& White 100\%\\ \hline
BLACk											& X					& X						& X					& X					& Black\\ \hline
SDICheck									& X					& X						& X					& X					& SDI Check Field\\ \hline
DTIMing										& X					& X						& X					& X					& Digital timing\\ \hline
FDTest										& X					& X						& X					& X					& Field Delay test\\ \hline
BOWTie										& X					& X						& X					& X					& Bow Tie\\ \hline
ABLanking									& X					& X						& X					& X					& Analog Blanking\\ \hline
DGRey											& X					& X						& X					& X					& Digital Grey\\ \hline
FSWave										& X					& X						& X					& X					& Field Square wave\\ \hline
BLWH01										&						&							& X					& X					& 0.1 Hz Bl/Wh\\ \hline
EOLine										&						&							& X					& X					& End of line\\ \hline
WEOLine 									&						&							& X					& X					& White end of line\\ \hline
BEOLine										&						&							& X					& X					& Blue end of line\\ \hline
REOLine										&						&							& X					& X					& Red end of line\\ \hline
YEOLine										&						&							& X					& X					& Yellow end of line\\ \hline
CEOLine										&						&							& X					& X					& Cyan end of line\\ \hline
SRAMP											& X					& X						& X					& X					& Shallow ramp\\ \hline
RAMP											& X					& X						& X					& X					& Ramp\\ \hline
LRAMp											& X					& X						& X					& X					& Limit Ramp\\ \hline
VRAMp											& X					& X						& X					& X					& Valid Ramp\\ \hline
CCIR17										&						&							&						& X					& CCIR line 17\\ \hline
CCIR18										&						&							&						& X					& CCIR line 18\\ \hline
CCIR330										&						&							&						& X					& CCIR line 330\\ \hline
CCIR331										&						&							&						& X					& CCIR line 331\\ \hline
STAircase5								& X					& X						& X					& X					& Staircase 5 step\\ \hline
MSTaircase5								& X					& X						& X					& X					& Staircase 5 step, modulated\\ \hline
STAircase10								&						&							& X					& X					& Staircase 10 step\\ \hline
PBAR											& X					& X						& X					& X					& Pulse \& Bar\\ \hline
YGRamp										&						&							& X					& X					& Ramp Yellow/Grey\\ \hline
GBRamp										&						&							& X					& X					& Ramp Grey/Blue\\ \hline
CGRamp										&						&							& X					& X					& Ramp Cyan/Grey\\ \hline
GRRamp										&						&							& X					& X					& Ramp Grey/Red\\ \hline
CBYCramp									&						&							& X					& X					& Ramp Cb, CR, Y\\ \hline
PHILips43									&						&							& X					& X					& Philips pattern in 4:3 format\\ \hline
PHILips169								&						&							& X					& X					& Philips pattern in 16:9 format\\ \hline
FUBK43										&						&							& X					& X					& FuBK pattern in 4:3 format\\ \hline
FUBK169										&						&							& X					& X					& FuBK pattern in 16:9 format\\ \hline
CROSshatch								& X					& X						& X					& X					& Cross hatch\\ \hline
PLUGe											& X					& X						& X					& X					& Pluge\\ \hline
SAFarea										& X					& X						& X					& X					& Safe area\\ \hline
VNT01											&						&							& X					& X					& VMT01\\ \hline
\end{tabular}
\end{center}

\textbf{Note:} Not all the patterns are available in both systems. Trying to select a pattern not available in the active system will result in an error, namely: -200,``Execution error'' 

Example:\\
\textit{OUTP:STPG2:PATT WIN15}\\
set the pattern in the STPG module to window 15\%\\
\textit{OUP:STPG2:PATT?}\\
response: WIN15

\textbf{OUTPUT:STPGenerator1:PATTern:MODify:APAL}\\
\textbf{OUTPUT:STPGenerator1:PATTern:MODify:PLUGe}\\
\textbf{OUTPUT:STPGenerator1:PATTern:MODify:STAircase10}\\
\textbf{OUTPUT:STPGenerator1:PATTern:MODify:MOTion}

\textbf{OUTPUT:STPGenerator2:PATTern:MODify:APAL}\\
\textbf{OUTPUT:STPGenerator2:PATTern:MODify:PLUGe}\\
\textbf{OUTPUT:STPGenerator2:PATTern:MODify:STAircase10}\\
\textbf{OUTPUT:STPGenerator2:PATTern:MODify:MOTion}\\
\textbf{OUTPUT:STPGenerator2:PATTern:MODify:CIRCles}

\textbf{OUTPUT:STPGenerator5:PATTern:MODify:APAL}\\
\textbf{OUTPUT:STPGenerator5:PATTern:MODify:PLUGe}\\
\textbf{OUTPUT:STPGenerator5:PATTern:MODify:STAircase10}\\
\textbf{OUTPUT:STPGenerator5:PATTern:MODify:MOTion}\\
\textbf{OUTPUT:STPGenerator5:PATTern:MODify:CIRCles}\\
Commands to enable/disable the modifications of a complex pattern in an optional PT8632 and PT8633 SDI test Pattern Generator in the PT5230. The possible selections are: OFF and ON. 

\textbf{Note:} The above modifications are only available when a Philips or FuBK pattern has been selected. Trying to select a pattern will result in an error, namely: -200,``Execution error''

Example:\\
\textit{OUTP:STPG2:PATT:MOD:APAL OFF}\\
remove anti-PAL from a complex pattern in STPG module 2\\
\textit{OUP:STPG2:PATT: MOD:APAL?}\\
response: OFF

\textbf{OUTPut:STPGenerator1:TEXT:STRing1}\\
\textbf{OUTPut:STPGenerator1:TEXT:STRing2}\\
\textbf{OUTPut:STPGenerator1:TEXT:STRing3}

\textbf{OUTPut:STPGenerator2:TEXT:STRing1}\\
\textbf{OUTPut:STPGenerator2:TEXT:STRing2}\\
\textbf{OUTPut:STPGenerator2:TEXT:STRing3}

\textbf{OUTPut:STPGenerator5:TEXT:STRing1}\\
\textbf{OUTPut:STPGenerator5:TEXT:STRing2}\\
\textbf{OUTPut:STPGenerator5:TEXT:STRing3}\\
Command to insert one or more text strings into the pattern of the optional PT8632 and PT8633 SDI Test pattern Generator. Three parameters are possible, i.e. OFF, ON and some text, ``TEXT''. The string being edited depends upon the pattern selected. One group of patterns are the standard patterns, e.g. 75\% Red, Colourbar etc. and another group is the complex pattern which is the Philips 4:3 pattern. The standard patterns will have three lines of text available, while the complex pattern only have two lines of text.

\textbf{Note:} To switch the text on/off use the parameters: ON or OFF. To alter the actual text: use ``TEXT''. The text is limited to sixteen characters. % For a list of available characters, please refer to Appendix XXX.

Example:\\
\textit{OUTP:STPG2:TEXT:STR1 ``HI THERE!''}\\
set text line 1 in STPG2 to HI THERE!\\
\textit{OUTP:STPG2:TEXT:STR1 ON}\\
switch text in the pattern ON\\
\textit{OUTP: STPG2:TEXT:TEXT?}\\
response: ON,``HI THERE!''

\textbf{OUTPut:STPGenerator1:TEXT:STYLe}\\
\textbf{OUTPut:STPGenerator2:TEXT:STYLe}\\
\textbf{OUTPut:STPGenerator5:TEXT:STYLe}\\
Command to select how the text is to be inserted into the Philips 4:3 pattern in the optional PT8632 and PT8633 SDI Test Pattern generator. The possible selections are STANdard or COMPlex. When choosing the standard style, the two text lines will be placed in the lower right corner. When choosing the complex style, the text will be placed in the upper and lower text fields in the Philips pattern, and in the left and right text field for the FuBK pattern.

\textbf{Note:} This command is only available with the Philips or FuBK patterns. Attempting to use the command for any other pattern will result in an error, namely. -200, ``Execution error''.

Example:\\
\textit{OUTP:STPG2:TEXT:STYL COMP}\\
set text style in STPG2 to complex\\
\textit{OUTP: STPG2:TEXT:STYL?}\\
response: COMPLEX

\textbf{OUTPut:STPGenerator1:TEXT:CLOCk}\\
\textbf{OUTPut:STPGenerator2:TEXT:CLOCk}\\
\textbf{OUTPut:STPGenerator5:TEXT: CLOCk}\\
Command to insert time/date information into a pattern in the optional PT8632 or PT8633 SDI Test Pattern generators. The possible selections are:

\begin{tabular}{|l|l|}
\hline
\textbf{Signal:} & \textbf{Description:} \\ \hline
			& Description \\ \hline
OFF		& No time- or date-information\\ \hline
TIMe	& Time information\\ \hline
DTIMe	& Time- and date-information\\ \hline
\end{tabular}

\textbf{Note:} This command requires the optional PT8637 Time Clock Interface to be present.

Example:\\
\textit{OUTP:STPG1:TEXT:CLOC TIM}\\
insert time into the pattern in ATPG module1\\
\textit{OUTP: STPG1:TEXT:CLOC?}\\
response: TIME

\textbf{OUTPut:STPGenerator1:SYSTem}\\
\textbf{OUTPut:STPGenerator2:SYSTem}\\
\textbf{OUTPut:STPGenerator5:SYSTem}\\
Command to select the system of an optional PT 8632 and PT8633 SDI Test Pattern Generators. Systems available are:

\begin{tabular}{|l|l|}
\hline
System:	& Description: \\ \hline
SDI625	& 625/50 system \\ \hline
Sdi525	& 525/59.94 system \\ \hline
\end{tabular}

\textbf{Note:} If the pattern becomes invalid when selecting a new system, the pattern will change according to:

\begin{tabular}{|l@{$\rightarrow$}l|}
\hline
\multicolumn{2}{|l|}{625/50 specific patterns:} \\ \hline
EBU C.Bar 				& SMPTE C.Bar \\ \hline
75\% C.Bar+Grey 	& SMPTE C.Bar \\ \hline
75\% C.BAR+Red 		& SMPTE C.Bar \\ \hline
Philips 4:3 			& Crosshatch (only for PT8632) \\ \hline
FuBK 4:3 					& Philips 4:3 \\ \hline
FuBK 16:9 				& Philips 16:9 \\ \hline
VMT01 						& Crosshatch \\ \hline
\multicolumn{2}{|l|}{525/59.94 specific patterns:} \\ \hline
SMPTE C.Bar 			& EBU C.Bar \\ \hline
FCC C.Bar 				& EBU C.Bar \\ \hline
\end{tabular}

Example:\\
\textit{OUTP:STPG2:SYST SDI625}\\
set the system in STPG module to 625/50\\
\textit{OUTP: STPG2:SYST?}\\ response: SDI625

\textbf{OUTPut:STPGenerator1:EDHinsert}\\
\textbf{OUTPut:STPGenerator2:EDHinsert}\\
\textbf{OUTPut:STPGenerator5:EDHinsert}\\
Command to select the pattern of an optional PT 8632 or PT8633 SDI Test Pattern Generator in the PT5300 . Possible selections are ON or OFF.

Example:\\
\textit{OUTP:STPG2:EDH OFF}\\
set EDH insertion in STPG module 2 to OFF\\
\textit{OUTP: STPG2:EDH?}
response: OFF

\textbf{OUTPut:STPGenerator1:EMBaudio:SIGNal}\\
\textbf{OUTPut:STPGenerator2:EMBaudio:SIGNal}\\
\textbf{OUTPut:STPGenerator5:EMBaudio:SIGNal}\\
Command to select the signal of the embedded audio in an optional PT 8632 and PT8633 SDI Test Pattern Generator in the PT5300. Possible selections are:

\begin{tabular}{|l|l|}
\hline
System:		& Description: \\ \hline
OFF				& Off \\ \hline
S800HZ		& Stereo 800 Hz \\ \hline 
S1KHZ			& Stereo 1 kHz\\ \hline
SEBu1KHZ EBU	& Stereo 1kHz\\ \hline
SBBC1KHZ BBC	& Stereo 1 kHz\\ \hline
MEBu1KHZ EBU	& Mono 1 kHz\\ \hline
M1KHZ			& Mono 1 kHz\\ \hline
DUAL			& Dual Sound\\ \hline
\end{tabular}

Example:\\
\textit{OUTP:STPG1:EMB:SIGN S1KHZ}
set the embedded audio signal in STPG module 1 to stereo 1kHz\\
\textit{OUTP: STPG1:EMB:SIGN?}\\
response: S1KHZ

\textbf{OUTPut:STPGenerator1:EMBaudio:LEVel}\\
\textbf{OUTPut:STPGenerator2:EMBaudio:LEVel}\\
\textbf{OUTPut:STPGenerator5:EMBaudio:LEVel}\\
Command to select the level of the embedded audio in an optional PT 8632 and PT8633 SDI Test Pattern Generator in the PT5300. Possible selections are:

\begin{tabular}{|l|l|}
\hline
System:		& Description: \\ \hline
SILence		& Silence\\ \hline
DB0FS			& 0 dB\\ \hline
DB9FS			& -9 dB\\ \hline
DB12FS		& -12 dB\\ \hline
DB15FS		& -15 dB\\ \hline
DB16FS		& -16 dB\\ \hline
DB18FS		& -18 dB\\ \hline
DB20FS		& -20 dB\\ \hline
\end{tabular}

Example:\\
\textit{OUTP:STPG1:EMB:LEV DB0FS}\\
set the embedded audio signal in STPG module 1 to 0dB\\
\textit{OUTP: STPG1:EMB:LEV?}\\
response: DB0FS

\textbf{OUTPut:STPGenerator1:EMBaudio:GROup}\\
\textbf{OUTPut:STPGenerator2:EMBaudio GROup}\\
\textbf{OUTPut:STPGenerator5:EMBaudio: GROup}\\
Command to select the level of the embedded audio group in an optional PT8633 SDI Test Pattern Generator. Possible selections are:

\begin{tabular}{|l|l|}
\hline
Group:		& Description: \\ \hline
GROup1		& Audio Group 1\\ \hline
GROup2		& Audio Group 2\\ \hline
GROup3		& Audio Group 3\\ \hline
GROup4		& Audio Group 4\\ \hline
\end{tabular}

Example:\\
\textit{OUTP:STPG1:EMB:GRO GRO3}\\
set the embedded audio group in STPG module 1 to group3\\
\textit{OUTP: STPG1:EMB: GRO?}\\
response: GROUP3

\textbf{OUTPut:STPGenerator1:DELay}\\
\textbf{OUTPut:STPGenerator2:DELay}\\
\textbf{OUTPut:STPGenerator5:DELay}\\
Command to set the delay of an optional PT 8632 and PT8633 SDI Test Pattern Generator. The delay is defined by three parameters:
$<$Field$>$,$<$Line$>$,$<$HTime$>$, where $<$Field$>$ sets the field offset, $<$Line$>$ sets the line offset and $<$HTime$>$ sets the horizontal time in ns, i.e.
\begin{itemize}
\item HTime(PAL) $<$64000.0ns
\item HTime(NTSC) $<$63492.1ns
\end{itemize}

\textbf{Note:} It is not possible to select a delay outside the range of the selected system. See table below:

\begin{tabular}{|p{5em}|p{5em}|p{5em}|p{5em}|}
\hline
\multicolumn{4}{|c|}{Analogue} \\ 
\hline
\multicolumn{2}{|c|}{PAL, 625 Lines} & \multicolumn{2}{|c|}{NTSC, 625 Lines} \\ 
\hline
Field: 	& Line: 			& Field: 				& Line: 			\\ \hline
-0			& -0,..,-312	& -0						& -0,..,-262\\ \hline
+0			& +0,..,+311	& +0						& -0,..,+261\\ \hline
+1			& +0					& +1						& +0\\ \hline
\end{tabular}

Example:\\
\textit{OUTP:STPG1:DEL-0,-12,-148.0}
set the delay in STPG module 1 to -0 field, -12 line \& -148.0 ns\\
\textit{OUTP:STPG1:DeL?}\\ 
response:-0,-012,-00148.0

\textbf{OUTPut:STPGenerator1:VERSion?}\\
\textbf{OUTPut:STPGenerator2:VERSion?}\\
\textbf{OUTPut:STPGenerator5:VERSion?}\\
Command to display the version of an optional PT8632 or PT8633 SDI Test Pattern Generator. The response contains four fields:

\begin{itemize}
\item Field 1: Company name
\item Field 2: Type name
\item Field 3: Serial number (KUxxxxxx)
\item Field 4: Software version for the PT 8632 or PT8633 SDI Test Signal Generator
\end{itemize}

Example:\\
\textit{OUTP:STPG1:VERS?}\\
response: PTV,PT8632,KU123456,2.0

\textbf{OUTPut:STPGenerator1?}\\
\textbf{OUTPut:STPGenerator2?}\\
\textbf{OUTPut:STPGenerator5?}\\
Command to display the complete setting of an optional PT 8632 or PT8633 SDI Test Pattern Generator. The response contains ten fields: 

$<$Pattern$>$, $<$Text insertion$>$, $<$System$>$, $<$EDH$>$, $<$Audio signal$>$, $<$Audio level$>$, $<$Audio group$>$, $<$Field$>$, $<$Line$>$, $<$Ftime$>$. 

For an explanation of the response, see the commands: 

OUTP:STPGn:PATT, OUTP:STPGn:TEXT, OUTP:STPGn:SYST, OUTP:STPGn:EDH, OUTP:STPGn:EMB:SIGN, OUTP:STPGn:EMB:LEV, OUTP:STPGn:EMB:GRO and OUTP:STPGn:Del, where n: 1, 2 or 5

Example:\\
\textit{OUTP:STPG2?}\\
response:\\
CBEBU,OFF,``DIGITAL'',SDI625,OFF,DUAL,DBM9FS GROUP1,+0,+001,+12345.5

\textbf{OUTPut:AUDio1:SIGNal}\\
\textbf{OUTPut:AUDio2:SIGNal}\\
Command to select the audio signal in an optional PT 8635 Dual AES/EBU Audio Generator. Possible selections are:

\begin{tabular}{|l|l|}
\hline
System:		& Description: \\ \hline
S800HZ		& Stereo 800 Hz\\ \hline
S1KHZ			& Stereo 1 kHz\\ \hline
SEBu1KHZ	& EBU Stereo 1kHz\\ \hline
SBBC1KHZ	& BBC Stereo 1 kHz\\ \hline
MEBu1KHZ	& EBU Mono 1 kHz\\ \hline
M1KHZ			& Mono 1 kHz\\ \hline
DUAL			& Dual Sound\\ \hline
F48KHZ		& Wordclock (48 kHz)\\ \hline
\end{tabular}

Example:\\
\textit{OUTP:AUD1:SIGN DUAL}\\
set the audio signal in the generator to dual sound\\
\textit{OUTP:AUD1:SIGN?}\\
response: DUAL

\textbf{OUTPut:AUDio1:LEVel}\\
Command to select the audio level in an optional PT 8635 Dual AES/EBU Audio Generator. Possible selections are:

\begin{tabular}{|l|l|}
\hline
Signal:		& Description: \\ \hline
SILence		& Silence \\ \hline
DB0FS			& 0 dB \\ \hline
DB9FS			& -9 dB \\ \hline
DB12FS		& -12 dB \\ \hline
DB15FS		& -15 dB \\ \hline
DB16FS		& -16 dB \\ \hline
DB18FS		& -18 dB \\ \hline
DB20FS		& -20 dB \\ \hline
\end{tabular}

Example:\\
\textit{OUTP:AUD1:LEV DB20FS}\\
set the audio level in the generator to -20 dB\\
\textit{OUTP:AUD1:LEV?}\\
response: DB20FS

\textbf{OUTPut:AUDio1:TIMing}\\
Command to select the audio timing in an optional PT 8635 Dual AES/EBU Audio Generator. Possible selections are:

\begin{tabular}{|l|l|}
\hline
Signal:		& Description: \\ \hline
PAL				&	\\ \hline
NTSC1			& Phase AES0\\ \hline
NTSC2			& Phase AES1\\ \hline
NTSC3			& Phase AES2\\ \hline
NTSC4			& Phase AES3\\ \hline
NTSC5			& Phase AES4\\ \hline
\end{tabular}

Example:\\
\textit{OUTP:AUD1:TIM NTSC3}\\
set the audio timing in the generator to NTSC3
\textit{OUTP:AUD1:TIM?}\\
response: NTSC3

\textbf{OUTPut:AUDio1:VERSion?}\
Command to display the version of an optional PT 8635 Dual AES\&EBU Audio Generator. The response contains four fields:
\begin{itemize}
\item Field 1: Company name
\item Field 2: Type name
\item Field 3: Serial number (KUxxxxxx)
\item Field 4: Not available for this option, i.e. the returned value is 0.
\end{itemize}

Example:\\
\textit{OUTP:AUD1:VERS?}\\
response: PTV,PT8635,KU123456,2.0

\textbf{OUTPut:AUDio1?}\\
Command to display the complete settings of an optional PT 8635 Dual AES/EBU Audio Generator The response contains three fields:
$<$Signal$>$,$<$Level$>$,$<$Timing$>$. 

For an explanation of the response, see the commands: 

OUTP:AUDn:SIGN, OTUP:AUDn:LEV and OUTP:AUDn:TIM, where n is 1 or 2.

Example:\\
\textit{OUTP:AUD1?}\\
response: DUAL, SILENCE, NTSC3

\textbf{OUTPut:TIMeclock:DFORmat}\\
Command to set the date of an optional PT8637 Time Clock Interface in the PT5230. Possible selections are: DMY, MDY and YMD.

Example:\\
\textit{OUTP:TIM:DFOR MDY}\\
select displaying date as month, date, year\\
\textit{OUTP:TIM:DFOR?}\\
response: MDY

\textbf{OUTPut:TIMeclock:DATe}\\
Command to set the date of an optional PT8637 Time Clock Interface. The date must be entered as three numeric parameters separated by commas. The parameter must be entered as year, month, date. Entering an illegal date will result in error, namely: -200,``Execution error''

Example:\\
\textit{OUTP:TIM:DAT 05,12,2}\\
set the date to 2nd December 2005\\
\textit{OUTP:TIM:DAT?}\\
response: 05,12.2

\textbf{OUTPut:TIMeclock:TFORmat}\\
Command to set the date of an optional PT8637 Time Clock Interface. Possible selections are: HOUR12 and HOUR24. Entering an illegal date will result in error, namely: -200,``Execution error''

Example:\\
\textit{OUTP:TIM:TFOR HOUR12}\\
select 12 hour date format\\
\textit{OUTP:TIM:TFOR?}\\
response: HOUR12

\textbf{OUTPut:TIMeclock:TIMe}\\
Command to set the time of an optional PT8637 Time Clock Interface. The date must be entered as three numeric parameters separated by commas. The parameter must be entered as hour, minute, second. Entering an illegal time will result in error, namely: -200,``Execution error''

Example:\\
\textit{OUTP:TIM:TIM 08,34,12}\\
set time to 08:34:12, i.e. 34 minutes past 8 o'clock\\
\textit{OUTP:TIM:TIM?}\\
response: 8,34,12

\textbf{OUTPut:TIMeclock:REFerence}\\
Command to set the reference of an optional PT8637 Time Clock Interface. Possible selections are:

\begin{tabular}{|l|l|}
\hline
Signal:		& Description: \\ \hline
LTC				& LTC on XLR input\\ \hline
VITC			& VITC on genlock signal\\ \hline
VFFRequency	& Video Field frequency\\ \hline
REF1HZ		& 1 HZ pulse\\ \hline
INTernal	& Internal\\ \hline
\end{tabular}

Example:\\
\textit{OUTP:TIM:REF VITC}\\
selects VITC\\
\textit{OUTP:TIM: REF?}\\
response: VITC

\textbf{OUTPut:TIMeclock:OFFSet}\\
Command to set time offset of an optional PT8637 Time Clock Interface. 

Example:\\
\textit{OUTP:TIM:TIM:OFFS 0.3}\\
set time offset to +0.3 second\\
\textit{OUTP:TIM:OFFS?}\\
response: 0.3

\textbf{OUTPut:TIMeclock:VERSion?}\\
Command to display the version of an optional PT 86037 Time Code Interface. The response contains four fields:
\begin{itemize}
\item Field 1: Company name
\item Field 2: Type name
\item Field 3: Serial number (KUxxxxxx)
\item Field 4: Not available for this option, i.e. the returned value is 0.
\end{itemize}

Example:\\
\textit{OUTP:TIM:VERS?}\\
response: PTV,PT8637,KU123456,0

\textbf{OUTPut:TIMeclock?}\\
Command to display the complete setting of an optional PT 8637 Time Clock Interface. The response contains six (ten) fields:
$<$Date format$>$,$<$Date$>$,$<$Time format$>$,$<$Time$>$,$<$Reference$>$,$<$Offset$>$. 

For an explanation of the response, see the commands: 

OUTP:TIM:DFOR, OUTP:TIM:DAT, OUTP:TIM:TFOR, OUTP:TIM:TIM, OUTP:TIM:REF, OUTP:TIM:OFFS.

\textbf{Note:} Due to that the response of date and time returns 3 values (both), then the response contains 10 fields.

Example:\\
\textit{OUTP:TIM:?}
response: MDY,12,12,98,HOUR24,8,0,0,LTC,0







\textbf{:OUTPut:TLG1:SYSTem}\\
\textbf{:OUTPut:TLG2:SYSTem}\\
\textbf{:OUTPut:TLG3:SYSTem}\\
Command to select the system of an optional PT8611 Tri-Level Sync Generators. Available systems are:

\begin{tabular}{|l|l|}
\hline
		OFF         &   OFF            \\ \hline
		HD1080P60   &   HD 1080P/60    \\ \hline
		HD1080P5994 &   HD 1080P/59.94 \\ \hline
    HD1080P50   &   HD 1080P/50    \\ \hline
    HD1080I30   &   HD 1080I/30    \\ \hline                   
    HD1080I2997 &   HD 1080I/29.97 \\ \hline                  
    HD1080I25   &   HD 1080I/25    \\ \hline                   
    HD1080P30   &   HD 1080P/30    \\ \hline                   
    HD1080P2997 &   HD 1080P/29.97 \\ \hline                  
    HD1080P25   &   HD 1080P/25    \\ \hline                   
    HD1080P24   &   HD 1080P/24    \\ \hline                   
    HD1080P2398 &   HD 1080P/23.98 \\ \hline                  
    HD1080sF30  &   HD 1080sF/30   \\ \hline                       
    HD1080sF2997&   HD 1080sF/29.97\\ \hline                      
    HD1080sF25  &   HD 1080sF/25   \\ \hline                       
    HD1080sF24  &   HD 1080sF/24   \\ \hline                       
    HD1080sF2398&   HD 1080sF/23.98\\ \hline                      
    HD720P60    &   HD 720P/60     \\ \hline                   
    HD720P5994  &   HD 720P/59.94  \\ \hline                  
    HD720P50    &   HD 720P/50     \\ \hline               
    HD720P30    &   HD 720P/30     \\ \hline               
    HD720P2997  &   HD 720P/29.97  \\ \hline              
    HD720P25    &   HD 720P/25     \\ \hline               
    HD720P24    &   HD 720P/24     \\ \hline               
    HD720P2398  &   HD 720P/23.98  \\ \hline
\end{tabular}

example:\\
\textit{:outp:tlg5:syst HD1080sF2398;}\\
sets system to HD 1080sF/23.98 

\textit{:outp:tlg5:syst?;}\\
response: HD1080sF2398

\textbf{:OUTPut:TLG1:DELay}\\
\textbf{:OUTPut:TLG2:DELay}\\
\textbf{:OUTPut:TLG3:DELay}\\
Command to set the delay of a PT8611 Tri-Level Sync Generator. The delay is defined by three parameters:

\begin{tabular}{l}
    $<$Field$>,<$Line$>,<$HTime$>$ \\
\end{tabular}

where $<$Field$>$ is the field offset, $<$Line$>$ is the line offset and $<$HTime$>$ sets the horizontal time in ns.

\textbf{Note:} It is not possible to select a delay outside the range for the selected system. See table below:

\begin{tabular}{|l|l|l|l|}
\hline
system      &       SYSTem    &      MIN     &        MAX \\ \hline
HD 1080P/60     &   HD1080P60   & -0,-562,0.0 & 0,562,14808.1\\ \hline
HD 1080P/59.94  &   HD1080P5994 & -0,-562,0.0 & 0,562,14822.9\\ \hline
HD 1080P/50     &   HD1080P50   & -0,-562,0.0 & 0,562,17771.0\\ \hline
HD 1080I/30     &   HD1080I30   & -0,-562,0.0 & 0,562,29622.9\\ \hline
HD 1080I/29.97  &   HD1080I2997 & -0,-562,0.0 & 0,562,29652.4\\ \hline
HD 1080I/25     &   HD1080I25   & -0,-562,0.0 & 0,562,35548.8\\ \hline
HD 1080P/30     &   HD1080P30   & -0,-562,0.0 & 0,562,29622.9\\ \hline
HD 1080P/29.97  &   HD1080P2997 & -0,-562,0.0 & 0,562,29652.4\\ \hline
HD 1080P/25     &   HD1080P25   & -0,-562,0.0 & 0,562,35548.8\\ \hline
HD 1080P/24     &   HD1080P24   & -0,-562,0.0 & 0,562,37030.3\\ \hline
HD 1080P/23.98  &   HD1080P2398 & -0,-562,0.0 & 0,562,37067.2\\ \hline
HD 1080sF/30    &   HD1080sF30  & -0,-562,0.0 & 0,562,29622.9\\ \hline
HD 1080sF/29.97 &   HD1080sF2997& -0,-562,0.0 & 0,562,29652.4\\ \hline
HD 1080sF/25    &   HD1080sF25  & -0,-562,0.0 & 0,562,35548.8\\ \hline
HD 1080sF/24    &   HD1080sF24  & -0,-562,0.0 & 0,562,37030.3\\ \hline
HD 1080sF/23.98 &   HD1080sF2398& -0,-562,0.0 & 0,562,37067.2\\ \hline
HD 720P/60      &   HD720P60    & -0,-374,0.0 & 0,375,22215.5\\ \hline
HD 720P/59.94   &   HD720P5994  & -0,-374,0.0 & 0,375,22237.7\\ \hline
HD 720P/50      &   HD720P50    & -0,-374,0.0 & 0,375,26659.9\\ \hline
HD 720P/30      &   HD720P30    & -0,-374,0.0 & 0,375,44437.7\\ \hline
HD 720P/29.97   &   HD720P2997  & -0,-374,0.0 & 0,375,44482.0\\ \hline
HD 720P/25      &   HD720P25    & -0,-374,0.0 & 0,375,53326.6\\ \hline
HD 720P/24      &   HD720P24    & -0,-374,0.0 & 0,375,55548.8\\ \hline
HD 720P/23.98   &   HD720P2398  & -0,-374,0.0 & 0,375,55604.2\\ \hline
\end{tabular}

example:\\
\textit{:OUTPut:TLG5:del 0,1,144.0;}\\
sets delay to 1 line and 144.0 ns

\textit{:OUTPut:TLG5:del?;}\\
response: +0,+001,+00141.4 (rounded)

\textbf{:OUTPut:HD1:PATTern}
\textbf{:OUTPut:HD2:PATTern}
\textbf{:OUTPut:HD3:PATTern}

Command to select the pattern of an optional PT8612 HD Test Pattern Generators. Patterns available are:
 
\begin{itemize}
\item BLACk
\item SDICheck
\item PLUGe
\item LRAMp
\item CLAPperbrd
\item COLOrbar
\item COMBInation
\item WINdow
\item CROSshatch
\item WHITe
\end{itemize}

example:\\
\textit{OUTP:HD2:PATT BLACK}\\
sets the pattern in the HD module BLACK

\textit{OUTP:HD2:PATT?}\\
response: BLACK

Some patterns have modifications.

Patterns:

\begin{itemize}
\item COLOrbar
\item COMBInation
\end{itemize}

have the following modifications:

\begin{itemize}
\item HH  - 100/0/100/0
\item HS  - 100/0/75/0
\item SS  - 75/0/75/0
\end{itemize}

Patterns:

\begin{itemize}
\item WINdow
\item WHITe
\end{itemize}

have the following modifications:

\begin{tabular}{|l|l|}
\hline
AM5   &    -5\% White    \\ \hline
A0    &     0\% White    \\ \hline
A5    &     5\% White    \\ \hline
A10   &    10\% White    \\ \hline
A15   &    15\% White    \\ \hline
A20   &    20\% White    \\ \hline
A25   &    25\% White    \\ \hline
A30   &    30\% White    \\ \hline
A35   &    35\% White    \\ \hline
A40   &    40\% White     \\ \hline
A45   &    45\% White     \\ \hline
A50   &    50\% White     \\ \hline
A55   &    55\% White     \\ \hline
A60   &    60\% White     \\ \hline
A65   &    65\% White     \\ \hline
A70   &    70\% White     \\ \hline
A75   &    75\% White     \\ \hline
A80   &    80\% White     \\ \hline
A85   &    85\% White     \\ \hline
A90   &    90\% White     \\ \hline
A95   &    95\% White     \\ \hline
A100  &   100\% White    \\ \hline
A105  &   105\% White    \\ \hline
\end{tabular}

example:\\

\textit{:outp:dl5:patt:mod ss;}\\
sets pattern modification to 75/0/75/0 for patterns COLORBAR and COMBINATION

\textit{:outp:dl5:patt:mod?;}\\
response: SS

\textit{:outp:dl5:patt:mod AM5;}\\
sets pattern modification to -5\% white level for patterns WINDOW and WHITE

\textit{:outp:dl5:patt:mod?;}\\
response: AM5

Trying to select modification not available for given pattern will result in an error, namely -224,``Illegal parameter value''.

trying to select modification for non-modifiable patterns results in: -200,``Execution error''.

\textbf{:OUTPut:HD1:SYSTem}\\
\textbf{:OUTPut:HD2:SYSTem}\\
\textbf{:OUTPut:HD3:SYSTem}\\
Command to select the system of an optional PT8612 HD Test Pattern Generators.  Available systems are:

\begin{tabular}{|l|l|}
\hline
OFF          &   OFF                  \\ \hline
HD1080I30    &   HD 1080I/30          \\ \hline
HD1080I2997  &   HD 1080I/29.97       \\ \hline
HD1080I25    &   HD 1080I/25          \\ \hline
HD1080P30    &   HD 1080P/30          \\ \hline
HD1080P2997  &   HD 1080P/29.97       \\ \hline
HD1080P25    &   HD 1080P/25          \\ \hline
HD1080P24    &   HD 1080P/24          \\ \hline
HD1080P2398  &   HD 1080P/23.98       \\ \hline
HD720P60     &   HD 720P/60           \\ \hline
HD720P5994   &   HD 720P/59.94        \\ \hline
HD720P50     &   HD 720P/50           \\ \hline
HD720P30     &   HD 720P/30           \\ \hline
HD720P2997   &   HD 720P/29.97        \\ \hline
HD720P25     &   HD 720P/25           \\ \hline
HD720P24     &   HD 720P/24           \\ \hline
HD720P2398   &   HD 720P/23.98        \\ \hline
SD525        &   SD 487I/29.97 (525)  \\ \hline
SD625        &   SD 576I/25 (625)     \\ \hline
\end{tabular}

example:\\
\textit{:outp:dl5:syst HD1080P2398;}\\
sets system to HD 1080P/23.98

\textit{:outp:dl5:syst?;}\\
response: HD1080P2398

\textbf{:OUTPut:HD1:EMBaudio:SIGnal}\\
\textbf{:OUTPut:HD2:EMBaudio:SIGnal}\\
\textbf{:OUTPut:HD3:EMBaudio:SIGnal}\\
Command to select the signal of the embedded audio on PT8612 HD Test Pattern Generators.  Available signals are:

\begin{tabular}{|l|l|}
\hline
SILence  &   Silence				\\ \hline
SINE     &   1 kHz sine				\\ \hline
CLICK    &   1 kHz sine with click	\\ \hline
OFF      &   No embedded audio\\ \hline
\end{tabular}

example:\\
\textit{:outp:dl5:emb:sign off;}\\
sets embedded audio to off

\textit{:outp:dl5:emb:sign?;}\\
response: OFF

\textbf{:OUTPut:HD1:EMBaudio:LEVel}\\
\textbf{:OUTPut:HD3:EMBaudio:LEVel}\\
\textbf{:OUTPut:HD4:EMBaudio:LEVel}\\
Command to select the level of the embedded audio on PT8612 HD Test Pattern Generators.  Available levels are:

\begin{tabular}{|l|l|}
\hline
DB0FS  &    0 dB Full Scale\\ \hline
DB6FS  &   -6 dB Full Scale\\ \hline
DB12FS &  -12 dB Full Scale \\ \hline
DB18FS &  -18 dB Full Scale \\ \hline
DB24FS &  -24 dB Full Scale \\ \hline
\end{tabular}

example:\\
\textit{:outp:dl5:emb:level DB12FS;}\\
sets embedded signal level to -12 dB Full Scale

\textit{:outp:dl5:emb:lev?;}\\
response: DB12FS

\textbf{:OUTPut:HD1:EMBaudio:CLIck}\\
\textbf{:OUTPut:HD3:EMBaudio:CLIck}\\
\textbf{:OUTPut:HD6:EMBaudio:CLIck}\\
Command to select the click timing of the embedded audio on PT8612 HD Test Pattern Generators.  Available values are (in milliseconds):

\begin{tabular}{l}
    -499 \ldots 500 \\
\end{tabular}

example:\\
\textit{:outp:dl5:EMBaudio:CLIck -200;}\\
sets embedded audio click to -200 ms

\textit{:outp:dl5:emb:cli?;}\\
response: -200

\textbf{:OUTPut:HD1:TEXT:STRing1}\\
\textbf{:OUTPut:HD4:TEXT:STRing2}\\
\textbf{:OUTPut:HD5:TEXT:STRing3}\\
Command to insert one or more (up to 3) text strings into the pattern on PT8612 HD Test Pattern Generators. Three parameters are possible, i.e. OFF, ON and some text, ``Text''.

\textbf{Note:} To switch the text on/off use the parameters ON or OFF. To alter the actual text use ``TEXT''. The text is limited to sixteen characters. Only printable members of standard ASCII character set (7-bit) are available.  Using other values will result in -360,``Communication error''.

example:\\
\textit{:OUTPut:HD5:TEXT:STR3 ``HI THERE'';}\\
sets text line 3 in HD5 to ``HI THERE''

\textit{:OUTPut:HD5:TEXT:STR3 ON;}\\
switch text line 3 ON

\textit{:OUTPut:HD5:TEXT:str3?;}\\
response: ON,``HI THERE''

\textbf{:OUTPut:HD1:TEXT:MOVement}\\
\textbf{:OUTPut:HD4:TEXT:MOVement}\\
\textbf{:OUTPut:HD5:TEXT:MOVement}\\
Command to set movement of text on PT8612 HD Test Pattern Generators. Available possibilities are:

\begin{tabular}{|l|l|}
\hline
OFF       &   Text is stationary\\ \hline
VERtical  &   Vertical movement\\ \hline
HORizontal&   Horizontal movement\\ \hline
BOTH      &   Movement in both directions \\ \hline
\end{tabular}

example:\\
\textit{:OUTPut:HD5:TEXT:mov both;}\\
sets text movement to both directions

\textit{:OUTPut:HD5:TEXT:mov?;}\\
response: BOTH

\textbf{:OUTPut:HD1:TEXT:SCAle}\\
\textbf{:OUTPut:HD4:TEXT:SCAle}\\
\textbf{:OUTPut:HD5:TEXT:SCAle}\\
Command to set text scale on PT8612 HD Test Pattern Generators. Available possibilities are: 1, 2, 3 and 4.

example:\\
\textit{:OUTPut:HD5:TEXT:sca 3;}\\
sets text scale to 3

\textit{:OUTPut:HD5:TEXT:sca?;}\\
response: 3

\textbf{:OUTPut:HD1:TEXT:COLor}\\
\textbf{:OUTPut:HD4:TEXT:COLor}\\
\textbf{:OUTPut:HD5:TEXT:COLor}\\

Command to set color of text on PT8612 HD Test Pattern Generators. Available colors are:

\begin{itemize}
\item WHIte \\
\item YELlow\\
\item CYAn\\
\item GREen\\
\item MAGenta\\
\item BLUe\\
\item BLAck\\
\end{itemize}

example:\\
\textit{:OUTPut:HD5:TEXT:col mag;}\\
sets text color to magenta

\textit{:OUTPut:HD5:TEXT:color?;}\\
response: MAGENTA

\textbf{:OUTPut:HD1:TEXT:BACKground}\\
\textbf{:OUTPut:HD4:TEXT:BACKground}\\
\textbf{:OUTPut:HD5:TEXT:BACKground}\\
Command to set background color of text on PT8612 HD Test Pattern Generators. Available colors are:

\begin{itemize}
\item WHIte \\
\item YELlow\\
\item CYAn\\
\item GREen\\
\item MAGenta\\
\item BLUe\\
\item BLAck\\
\end{itemize}

example:\\
\textit{:OUTPut:HD5:TEXT:back mag;}\\
sets text background color to magenta

\textit{:OUTPut:HD5:TEXT:background?;}\\
response: MAGENTA

\textbf{:OUTPut:HD1:DELay}\\
\textbf{:OUTPut:HD4:DELay}\\
\textbf{:OUTPut:HD5:DELay}\\
Command to set the delay of a PT8612 HD Test Pattern Generator. The delay is defined by three parameters:

\begin{tabular}{l}
    $<$Field$>,<$Line$>,<$HTime$>$ \\
\end{tabular}

where $<$Field$>$ is the field offset, $<$Line$>$ is the line offset and $<$HTime$>$ sets the horizontal time in ns

\textbf{Note:} It is not possible to select a delay outside the range for the selected system.  See table below.

\begin{tabular}{|l|l|l|l|}
\hline
system         &     SYSTem     &     MIN      &       MAX\\ \hline
\hline
HD 1080I/30        &  HD1080I30    & -0,-562,0.0 & 0,562,29622.9\\ \hline
HD 1080I/29.97     &  HD1080I2997  & -0,-562,0.0 & 0,562,29652.4\\ \hline
HD 1080I/25        &  HD1080I25    & -0,-562,0.0 & 0,562,35548.8\\ \hline
HD 1080P/30        &  HD1080P30    & -0,-562,0.0 & 0,562,29622.9\\ \hline
HD 1080P/29.97     &  HD1080P2997  & -0,-562,0.0 & 0,562,29652.4\\ \hline
HD 1080P/25        &  HD1080P25    & -0,-562,0.0 & 0,562,35548.8\\ \hline
HD 1080P/24        &  HD1080P24    & -0,-562,0.0 & 0,562,37030.3\\ \hline
HD 1080P/23.98     &  HD1080P2398  & -0,-562,0.0 & 0,562,37067.2\\ \hline
HD 720P/60         &  HD720P60     & -0,-374,0.0 & 0,375,22215.5\\ \hline
HD 720P/59.94      &  HD720P5994   & -0,-374,0.0 & 0,375,22237.7\\ \hline
HD 720P/50         &  HD720P50     & -0,-374,0.0 & 0,375,26659.9\\ \hline
HD 720P/30         &  HD720P30     & -0,-374,0.0 & 0,375,44437.7\\ \hline
HD 720P/29.97      &  HD720P2997   & -0,-374,0.0 & 0,375,44482.0\\ \hline
HD 720P/25         &  HD720P25     & -0,-374,0.0 & 0,375,53326.6\\ \hline
HD 720P/24         &  HD720P24     & -0,-374,0.0 & 0,375,55548.8\\ \hline
HD 720P/23.98      &  HD720P2398   & -0,-374,0.0 & 0,375,55604.2\\ \hline
SD 487I/29.97 (525)&  SD525        & -0,-262,0.0 & 0,262,63548.8\\ \hline
SD 576I/25 (625)   &  SD625        & -0,-312,0.0 & 0,312,63993.3\\ \hline
\end{tabular}

example:\\
\textit{:OUTPut:HD5:del 0,1,144.0;}\\
sets delay to 1 line and 144.0 ns

\textit{:OUTPut:HD5:del?;}\\ 
response: +0,+001,+00141.4 (rounded)

\textit{:OUTPut:HD5:del -0,-561,-144.0;}\\
sets delay to -561 lines and 144.0 ns

\textit{:OUTPut:HD5:del?;}\\
response: -0,-561,-00141.4 (rounded)

\textit{:OUTPut:HD5:del -0,-562,0.0;}\\
sets delay to -562 lines and 0.0 ns

\textit{:OUTPut:HD5:del?;}\\
response: -0,-562,-00000.

\textbf{:OUTPut:HD1?}\\
\textbf{:OUTPut:HD4?}\\
\textbf{:OUTPut:HD5?}\\
Command to display complete setting of the PT8612 HD Test Pattern Generator.  The response contains 8 fields:

$<$Pattern$>,<$Text$>,<$System$>,<$Audio signal$>,<$Audio level$>,<$Field$>,<$Line$>,<$HTime$>$

\textbf{:OUTPut:DL1:PATTern}\\
\textbf{:OUTPut:DL2:PATTern}\\
\textbf{:OUTPut:DL3:PATTern}\\
Command to select the pattern of an optional PT8613 Dual Link HD Test Pattern Generators. Patterns available are:
  
\begin{itemize}
\item BLACk
\item SDICheck
\item PLUGe
\item LRAMp
\item CLAPperbrd
\item COLOrbar
\item COMBInation
\item WINdow
\item CROSshatch
\item WHITe
\end{itemize}

example:\\
\textit{OUTP:DL2:PATT BLACK}\\
sets the pattern in the DL module BLACK

\textit{OUTP:DL2:PATT?}\\
response: BLACK

Some patterns have modifications.

Patterns:

\begin{itemize}
\item COLOrbar
\item COMBInation
\end{itemize}

have the following modifications:

\begin{tabular}{l @{ - } l}
HH  & 100/0/100/0\\
HS  & 100/0/75/0\\
SS  & 75/0/75/0 \\
\end{tabular}

Patterns:

\begin{itemize}
\item WINdow
\item WHITe
\end{itemize}

have the following modifications:

\begin{tabular}{|l|l|}
\hline
AM5   &    -5\% White    \\ \hline
A0    &     0\% White    \\ \hline
A5    &     5\% White    \\ \hline
A10   &    10\% White    \\ \hline
A15   &    15\% White    \\ \hline
A20   &    20\% White    \\ \hline
A25   &    25\% White    \\ \hline
A30   &    30\% White    \\ \hline
A35   &    35\% White    \\ \hline
A40   &    40\% White     \\ \hline
A45   &    45\% White     \\ \hline
A50   &    50\% White     \\ \hline
A55   &    55\% White     \\ \hline
A60   &    60\% White     \\ \hline
A65   &    65\% White     \\ \hline
A70   &    70\% White     \\ \hline
A75   &    75\% White     \\ \hline
A80   &    80\% White     \\ \hline
A85   &    85\% White     \\ \hline
A90   &    90\% White     \\ \hline
A95   &    95\% White     \\ \hline
A100  &   100\% White    \\ \hline
A105  &   105\% White    \\ \hline
\end{tabular}  

example:\\
\textit{:outp:dl5:patt:mod ss;}\\
sets pattern modification to 75/0/75/0 for patterns COLORBAR and COMBINATION

\textit{:outp:dl5:patt:mod?;}\\
response: SS

\textit{:outp:dl5:patt:mod AM5;}\\
sets pattern modification to -5\% white level for patterns WINDOW and WHITE

\textit{:outp:dl5:patt:mod?;}\\
response: AM5

Trying to select modification not available for given pattern will result in an error, namely -224,``Illegal parameter value''. trying to select modification for non-modifiable patterns results in: -200,``Execution error''.

\textbf{:OUTPut:DL1:SYSTem}\\
\textbf{:OUTPut:DL2:SYSTem}\\
\textbf{:OUTPut:DL3:SYSTem}\\
Command to select the system of an optional PT8613 Dual Link HD Test Pattern Generators.  Available systems are:

\begin{tabular}{|l|l|}
\hline
OFF           &  OFF             \\ \hline
HD1080I30     &  HD 1080I/30     \\ \hline
HD1080I2997   &  HD 1080I/29.97  \\ \hline          
HD1080I25     &  HD 1080I/25     \\ \hline
HD1080P30     &  HD 1080P/30     \\ \hline
HD1080P2997   &  HD 1080P/29.97  \\ \hline
HD1080P25     &  HD 1080P/25     \\ \hline
HD1080P24     &  HD 1080P/24     \\ \hline
HD1080P2398   &  HD 1080P/23.98  \\ \hline
HD1080SF30    &  HD 1080sF/30    \\ \hline
HD1080SF2997  &  HD 1080sF/29.97 \\ \hline
HD1080SF25    &  HD 1080sF/25    \\ \hline
HD1080SF24    &  HD 1080sF/24    \\ \hline
HD1080SF2398  &  HD 1080sF/23.98 \\ \hline
HD720P60      &  HD 720P/60       \\ \hline
HD720P5994    &  HD 720P/59.94    \\ \hline
HD720P50      &  HD 720P/50       \\ \hline
HD720P30      &  HD 720P/30       \\ \hline
HD720P2997    &  HD 720P/29.97    \\ \hline
HD720P25      &  HD 720P/25       \\ \hline
HD720P24      &  HD 720P/24       \\ \hline
HD720P2398    &  HD 720P/23.98    \\ \hline
SD525         &  SD 487I/29.97 (525)\\ \hline
SD625         &  SD 576I/25 (625)\\ \hline
\end{tabular}

example:\\
\textit{:outp:dl5:syst HD1080SF2997;}\\
sets system to HD 1080sF/29.97

\textit{:outp:dl5:syst?;}\\
response: HD1080SF2997

\textbf{:OUTPut:DL1:SYSTem:INTERFace}\\
\textbf{:OUTPut:DL2:SYSTem:INTERFace}\\
\textbf{:OUTPut:DL3:SYSTem:INTERFace}\\

All HD 1080 systems have changeable interface.  The following interfaces are available:

\begin{tabular}{|l|l|l|l|l|}
\hline
I1 &  SINGLE& 4:2:2  & \YCBCR & 10-bit  \\ \hline
I2 &  DUAL  & 4:2:2:4& \YCBCR A& 12-bit  \\ \hline
I3 &  DUAL  & 4:4:4  & \YCBCR & 10-bit  \\ \hline
I4 &  DUAL  & 4:4:4:4& \YCBCR A& 12-bit  \\ \hline
I5 &  DUAL  & 4:4:4  & GBR   & 10-bit  \\  \hline
I6 &  DUAL  & 4:4:4:4& GBRA  & 12-bit  \\ \hline
\end{tabular}

example:\\
\textit{:outp:dl5:syst:INTERFace I6;}\\
sets interface  to DUAL 4:4:4:4 GBRA 12-bit

\textit{:outp:dl5:syst:INTERFace?;}\\
response: I6

\textbf{:OUTPut:DL1:EMBaudio:SIGnal}\\
\textbf{:OUTPut:DL2:EMBaudio:SIGnal}\\
\textbf{:OUTPut:DL3:EMBaudio:SIGnal}\\
Command to select the signal of the embedded audio on PT8613 Dual Link HD Test Pattern Generators.  Available signals are:

\begin{tabular}{|l|l|}
\hline
SILence  &   Silence\\ \hline
SINE     &   1 kHz sine\\ \hline
CLICK    &   1 kHz sine with click\\ \hline
OFF      &   No embedded audio \\ \hline
\end{tabular}

example:\\
\textit{:outp:dl5:emb:sign off;}\\
sets embedded audio to off

\textit{:outp:dl5:emb:sign?;}\\
response: OFF

\textbf{:OUTPut:DL1:EMBaudio:LEVel}\\
\textbf{:OUTPut:DL3:EMBaudio:LEVel}\\
\textbf{:OUTPut:DL4:EMBaudio:LEVel}\\
Command to select the level of the embedded audio on PT8613 Dual Link HD Test Pattern Generators.  Available levels are:

\begin{tabular}{|l|l|}
\hline
DB0FS  &    0 dB Full Scale\\ \hline
DB6FS  &   -6 dB Full Scale\\ \hline
DB12FS &  -12 dB Full Scale \\ \hline
DB18FS &  -18 dB Full Scale \\ \hline
DB24FS &  -24 dB Full Scale \\ \hline
\end{tabular}

example:\\
\textit{:outp:dl5:emb:level DB12FS;}\\
sets embedded signal level to -12 dB Full Scale

\textit{:outp:dl5:emb:lev?;}\\
response: DB12FS

\textbf{:OUTPut:DL1:EMBaudio:CLIck}\\
\textbf{:OUTPut:DL3:EMBaudio:CLIck}\\
\textbf{:OUTPut:DL6:EMBaudio:CLIck}\\
Command to select the click timing of the embedded audio on PT8613 Dual Link HD Test Pattern Generators.  Available values are (in milliseconds):

\begin{tabular}{l}
-499 \ldots 500\\
\end{tabular}

example:\\
\textit{:outp:dl5:EMBaudio:CLIck -200;}\\
sets embedded audio click to -200 ms

\textit{:outp:dl5:emb:cli?;}\\
response: -200

\textbf{:OUTPut:DL1:TEXT:STRing1}\\
\textbf{:OUTPut:DL4:TEXT:STRing2}\\
\textbf{:OUTPut:DL5:TEXT:STRing3}\\
Command to insert one or more (up to 3) text strings into the pattern on PT8613 Dual Link HD Test Pattern Generators.
Three parameters are possible, i.e. OFF, ON and some text, ``Text`''.

\textbf{Note:} To switch the text on/off use the parameters ON or OFF. To alter the actual text use ``TEXT''.  The text is limited to sixteen characters.  Only printable members of standard ASCII character set (7-bit) are available.  Using other values will result in -360,``Communication error''.

example:\\
\textit{:OUTPut:DL5:TEXT:STR3 ``HI THERE'';}\\
sets text line 3 in DL5 to ``HI THERE''

\textit{:OUTPut:DL5:TEXT:STR3 ON;}\\
switch text line 3 ON

\textit{:OUTPut:DL5:TEXT:str3?;}\\
response: ON,``HI THERE''

\textbf{:OUTPut:DL1:TEXT:MOVement}
\textbf{:OUTPut:DL4:TEXT:MOVement}
\textbf{:OUTPut:DL5:TEXT:MOVement}

Command to set movement of text on PT8613 Dual Link HD Test Pattern Generators. Available possibilities are:

\begin{tabular}{|l|l|}
\hline
OFF         & Text is stationary\\ \hline
VERtical    & Vertical movement\\ \hline
HORizontal  & Horizontal movement\\ \hline
%%BOTH        & Movement in both directions \\ \hline
\end{tabular}

example:\\
\textit{:OUTPut:DL5:TEXT:mov both; }\\
sets text movement to both directions

\textit{:OUTPut:DL5:TEXT:mov?;}\\
response: BOTH

\textbf{:OUTPut:DL1:TEXT:SCAle}\\
\textbf{:OUTPut:DL4:TEXT:SCAle}\\
\textbf{:OUTPut:DL5:TEXT:SCAle}\\

Command to set text scale on PT8613 Dual Link HD Test Pattern Generators. Available possibilities are: 1, 2, 3 and 4.

example:\\
\textit{:OUTPut:DL5:TEXT:sca 3;}\\
sets text scale to 3

\textbf{:OUTPut:DL5:TEXT:sca?; }\\
response: 3

\textbf{:OUTPut:DL1:TEXT:COLor}\\
\textbf{:OUTPut:DL4:TEXT:COLor}\\
\textbf{:OUTPut:DL5:TEXT:COLor}\\
Command to set color of text on PT8613 Dual Link HD Test Pattern Generators. Available colors are:

\begin{itemize}
\item BLACk
\item SDICheck
\item PLUGe
\item LRAMp
\item CLAPperbrd
\item COLOrbar
\item COMBInation
\item WINdow
\item CROSshatch
\item WHITe
\end{itemize}

example:\\
\textit{:OUTPut:DL5:TEXT:col mag;}\\
sets text color to magenta

\textit{:OUTPut:DL5:TEXT:color?;}\\     
response: MAGENTA

\textbf{:OUTPut:DL1:TEXT:BACKground}\\
\textbf{:OUTPut:DL4:TEXT:BACKground}\\
\textbf{:OUTPut:DL5:TEXT:BACKground}\\

Command to set background color of text on PT8613 Dual Link HD Test Pattern Generators. Available colors are:

\begin{itemize}
\item BLACk
\item SDICheck
\item PLUGe
\item LRAMp
\item CLAPperbrd
\item COLOrbar
\item COMBInation
\item WINdow
\item CROSshatch
\item WHITe
\end{itemize}

example:\\
\textit{:OUTPut:DL5:TEXT:back mag;}\\    
sets text background color to magenta

\textit{:OUTPut:DL5:TEXT:background?;}\\ 
response: MAGENTA

\textbf{:OUTPut:DL1:DELay}\\
\textbf{:OUTPut:DL4:DELay}\\
\textbf{:OUTPut:DL5:DELay}\\

Command to set the delay of a PT8613 Dual Link HD Test Pattern Generator. The delay is defined by three parameters:

\begin{tabular}{l}
$<$Field$>,<$Line$>,<$HTime$>$
\end{tabular}

where $<$Field$>$ is the field offset, $<$Line$>$ is the line offset and $<$HTime$>$ sets the horizontal time in ns

\textbf{Note:} It is not possible to select a delay outside the range for the selected system.  See table below.

\begin{tabular}{|l|l|l|l|}
\hline
system         &     SYSTem     &     MIN      &       MAX\\
\hline
HD 1080I/30        &  HD1080I30    & -0,-562,0.0 & 0,562,29622.9\\
HD 1080I/29.97     &  HD1080I2997  & -0,-562,0.0 & 0,562,29652.4\\
HD 1080I/25        &  HD1080I25    & -0,-562,0.0 & 0,562,35548.8\\
HD 1080P/30        &  HD1080P30    & -0,-562,0.0 & 0,562,29622.9\\
HD 1080P/29.97     &  HD1080P2997  & -0,-562,0.0 & 0,562,29652.4\\
HD 1080P/25        &  HD1080P25    & -0,-562,0.0 & 0,562,35548.8\\
HD 1080P/24        &  HD1080P24    & -0,-562,0.0 & 0,562,37030.3\\
HD 1080P/23.98     &  HD1080P2398  & -0,-562,0.0 & 0,562,37067.2\\
HD 720P/60         &  HD720P60     & -0,-374,0.0 & 0,375,22215.5\\
HD 720P/59.94      &  HD720P5994   & -0,-374,0.0 & 0,375,22237.7\\
HD 720P/50         &  HD720P50     & -0,-374,0.0 & 0,375,26659.9\\
HD 720P/30         &  HD720P30     & -0,-374,0.0 & 0,375,44437.7\\
HD 720P/29.97      &  HD720P2997   & -0,-374,0.0 & 0,375,44482.0\\
HD 720P/25         &  HD720P25     & -0,-374,0.0 & 0,375,53326.6\\
HD 720P/24         &  HD720P24     & -0,-374,0.0 & 0,375,55548.8\\
HD 720P/23.98      &  HD720P2398   & -0,-374,0.0 & 0,375,55604.2\\
SD 487I/29.97 (525)&  SD525        & -0,-262,0.0 & 0,262,63548.8\\
SD 576I/25 (625)   &  SD625        & -0,-312,0.0 & 0,312,63993.3\\
\hline
\end{tabular}

example:\\
\textit{:OUTPut:DL5:del 0,1,144.0;}\\
sets delay to 1 line and 144.0 ns

\textit{:OUTPut:DL5:del?;}\\           
response: +0,+001,+00141.4 (rounded)

\textit{:OUTPut:DL5:del -0,-561,-144.0;}\\
sets delay to -561 lines and 144.0 ns

\textit{:OUTPut:DL5:del?; }\\ 
response: -0,-561,-00141.4 (rounded)

\textit{:OUTPut:DL5:del -0,-562,0.0;}\\
sets delay to -562 lines and 0.0 ns

\textit{:OUTPut:DL5:del?;}\\  
response: -0,-562,-00000.

\textbf{:OUTPut:DL1?}\\
\textbf{:OUTPut:DL4?}\\
\textbf{:OUTPut:DL5?}\\

Command to display complete setting of the PT8613 Dual Link HD Test Pattern Generator.  The response contains 8 fields:

$<$Pattern$>,<$Text$>,<$System$>,<$Audio signal$>,<$Audio level$>,<$Field$>,<$Line$>,<$HTime$>$



\textbf{OUTPut:LTCG1:FORMat}\\
\textbf{OUTPut:LTCG2:FORMat}\\
Command to set the format of the LTC generator module. The delay is defined by four parameters: $<$Format$>,<$Syncmode$>,<$Hour$>,<$Min>
where $<$Format$>$ sets the format to one of the format listed in table \ref{table:LTC_formats}, $<$Syncmode$>$ sets the mode to reset the framecounter (see table \ref{table:sync_modes}), $<$Hour$>$ sets the sync hour and $<$Min$>$ sets the sync minute. The sync mode is only relevant for 29.97 frames per second, as other formats always stays synchronized to real time.

\begin{table}[hbt]
\centering

\begin{tabular}{|l|l|}
\hline
24FPS	& 24 Frames pr second \\ \hline
25FPS	&	25 Frames pr second \\ \hline
2997NOND & 29.97 Frames pr second, non drop frame \\ \hline
2997DROP & 29.97 Frames pr second, drop frame \\ \hline
30FPS & 30 Frames pr second \\ \hline
\end{tabular}

\caption{LTC Generator formats}
\label{table:LTC_formats}

\end{table}

\begin{table}[hbt]
\centering

\begin{tabular}{|l|l|}
\hline
NONE & No synchronization (Only manual sync.) \\ \hline
CONF & Confirm synchronization. Press \execute to confirm. \\ \hline
AUTO & Auto synchronization. No need to do anything. \\ \hline
\end{tabular}

\caption{LTC Generator Sync modes}
\label{table:sync_modes}

\end{table}

$<$Hour$>, <$Min$>$ sets the time, when the frame counter shall be resynced. The hour can be set to a number between 0 and 23. The minutes can be set between 0 and 59.

example:\\
\textit{:OUTPut:LTCG1:FORMAT '24FPS','NONE',0,0;}\\
Sets the format to 24 frames per second, the sync mode and time is not relevant for this mode.

\textit{:OUTPut:LTCG1:FORMAT '24FPS','NONE',0,0;}\\           
response: 24FPS,NONE,0,0

\textit{:OUTPut:LTCG1:FORMAT '2997DROPF','AUTO',23,30;}\\
Sets the format to 29.97 frames per second, drop frame. The frames counter will reset at 23:30, automatically.

\textbf{:OUTPut:LTCG1:OFFSET}\\
\textbf{:OUTPut:LTCG2:OFFSET}\\

Offsets the LTC generator $\pm$0.5 seconds relative to the absolute GPS time. Time is in nano seconds.

example:\\
\textit{:OUTPut:LTCG2:OFFSET 3000;}\\
Sets the LTC generator to have frame 0 start 3 micro seconds before the GPS time seconds.

\textit{:OUTPut:LTCG2:OFFSET?;}\\
response: 3000

\subsection{Communication Error Codes}

\subsubsection{Command Errors [-199, -100]}

\begin{longtable}{|l|p{25em}|}
\hline
Error			& Error string \\ 
Number		& [description/explanation/example] \\ \hline
-100			& Command error.  \\ \cline{2-2}
					& The command is invalid or incorrect. \\ \hline
-101			& Invalid character. \\ \cline{2-2}
					& A command or parameter contains an invalid character, e.g. a header containing an ampersand,
SYST:VERS\&\\ \hline
-102			& Syntax error. \\ \cline{2-2}
					& An unrecognized command or datatype was encountered, e.g. a string was received when the generator did not accept strings.\\ \hline
-103			& Invalid separator.\\ \cline{2-2}
					& A separator was expected, but an illegal character was encountered, e.g. the semicolon was omitted after a command, *IDN?:SYST:ERR?;\\ \hline
					
-104			& Data type error.\\ \cline{2-2}
					& A data element different than one allowed was encountered, e.g. numeric data was expected but string data was encountered. \\ \hline
-108			& Parameter not allowed.\\ \cline{2-2}
					& More parameters was received than expected for the command, e.g. the *IDN?; command accepts no parameters, so receiving *IDN? 2; is not allowed.\\ \hline
-109			& Missing parameter.\\ \cline{2-2}
					& Fewer parameters were received than expected for the command, e.g. OUTP:BB1:DEL2,2; is missing one parameter.\\ \hline
-110			& Command header error.\\ \cline{2-2}
					& An error was detected in the command header.\\ \hline
-111			& Header separator error.\\ \cline{2-2}
					& A character which is not a legal header separator was encountered, e.g. no white space followed the header, thus SYST:PRES:NAME``MACRO'' is an error.\\ \hline
-112			& Program mnemonic too long.\\ \cline{2-2}
					& The header contains more than twelve characters.\\ \hline
-113			& Undefined header.\\ \cline{2-2}
					& The header is syntactically correct, but is not defined for the device.\\ \hline
-114			& Header suffix out of range.\\ \cline{2-2}
					& The command is invalid because the value of the numeric suffix attached to the program mnemonic is out of range, e.g. OUTP:BB12? Is illegal because only 8 BB's exists.\\ \hline
-120			& Numeric data error.\\ \cline{2-2}
					& An error in the numeric data was encountered.\\ \hline
-121			& Invalid character in number.\\ \cline{2-2}
					& An invalid character for the data type was encountered, e.g. an alpha in a decimal value.\\ \hline
-123			& Exponent too large.\\ \cline{2-2}
					& The magnitude of the exponent was larger than 32000.\\ \hline
-124			& Too many digits.\\ \cline{2-2}
					& The mantissa of a decimal numeric data element contained more than 255 digits.\\ \hline
-128			& Numeric data not allowed.\\ \cline{2-2}
					& A legal numeric data was received, but the device does not accept one.\\ \hline
-130			& Suffix error.\\ \cline{2-2}
					& An error in the suffix was encountered.\\ \hline
-131			& Invalid suffix.\\ \cline{2-2}
					& The suffix is syntactically incorrect.\\ \hline
-134			& Suffix too long.\\ \cline{2-2}
					& The suffix contains more than twelve characters.\\ \hline
-138			& Suffix not allowed.\\ \cline{2-2}
					& A suffix was encountered after a numeric element, which does not allow suffixes.\\ \hline
-140			& Character data error.\\ \cline{2-2}
					& An error in the character was encountered.\\ \hline
-150			& String data error.\\ \cline{2-2}
					& An error in the string data was encountered.\\ \hline
-151			& Invalid string data.\\ \cline{2-2}
					& A string data element was expected, but was invalid for some reason, e.g. an END message was received before the terminal quote character.\\ \hline
-158			& String data not allowed.\\ \cline{2-2}
					& A string data element was received but was not allowed by the device.\\ \hline
-160			& Block data error.\\ \cline{2-2}
					& There is an error in the block data received.\\ \hline
-161			& Invalid block data.\\ \cline{2-2}
					& A block data was expected, but was invalid for some reason.\\ \hline
-170			& Expression error.\\ \cline{2-2}
					& There is an error in the expression received.\\ \hline
\end{longtable}

\subsubsection{Execution Errors [-299, -200]}
\begin{longtable}{|l|p{25em}|}
\hline
Error			& Error string \\ 
Number		& [description/explanation/example] \\ \hline
-200			& Execution error.\\ \hline
-220			& Parameter error.\\ \cline{2-2}
					& Indicates that a program data element related error occurred.\\ \hline
-222			& Data out of range.\\ \cline{2-2}
					& Indicates that a legal program data element was received but could not be executed because the interpreted values was outside the range as defined by the device, e.g. the command OUTP:BB1:SCHP 200; is illegal since the Sc-H Phase cannot exceed 180deg.\\ \hline
-223			& Too much data.\\ \cline{2-2}
					& Indicates that a legal program data element of block, expression, or string type was received that contained more data than the device could handle due to memory or related device-specific requirements.\\ \hline
-224			& Illegal parameter value.\\ \cline{2-2}
					& Used where exact value, from a list of possibles, was expected.\\ \hline
-233			& Invalid version.\\ \cline{2-2}
					& Indicates that a legal program data element was parsed but could not be executed because the version of the data is incorrect to the device.\\ \hline
-241			& Hardware missing.\\ \cline{2-2}
					& Indicates that a legal program command or query could not be executed because of missing device hardware.\\ \hline
\end{longtable}

\subsubsection{Device specific Errors [-399, -300]}
\begin{longtable}{|l|p{25em}|}
\hline
Error			& Error string \\ 
Number		& [description/explanation/example] \\ \hline
-300			& Device-specific error.\\ \hline
-350			& Queue overflow.\\ \hline
					& A specific code entered into the queue in lieu of the code that caused the error. This code indicates that there is no room in the queue and an error occurred but was not recorded.\\ \hline
-360			& Communication error.\\ \cline{2-2}
					& A communication error on the serial port was detected.\\ \hline
-361			& Parity error in program message.\\ \cline{2-2}
					& Parity bit not correct when data received on the serial port.\\ \hline
-362			& Framing error in program message.\\ \cline{2-2}
					& A stop bit was not detected when data was received, e.g. a bad rate mismatch.\\ \hline
-363			& Input buffer overrun.\\ \cline{2-2}
					& Software or hardware input buffer on serial port overflows.\\ \hline
\end{longtable}

\subsubsection{Query Errors [-499, -400]}
\begin{longtable}{|l|p{25em}|}
\hline
Error			& Error string \\ 
Number		& [description/explanation/example] \\ \hline
-400			& Query error.\\ \cline{2-2}
					& An error occurred during a query.\\ \hline
-410			& Query INTERRUPTED.\\ \cline{2-2}
					& Indicates that a condition causing an INTERRUPTED Query error occurred.\\ \hline
-420			& Query UNTERMINATED.\\ \cline{2-2}
					& Indicates that a condition causing an UNTERMINATED Query error occurred.\\ \hline
-430			& Query DEADLOCKED.\\ \cline{2-2}
					& Indicates that a condition causing a DEADLOCKED Query error occurred.\\ \hline
\end{longtable}

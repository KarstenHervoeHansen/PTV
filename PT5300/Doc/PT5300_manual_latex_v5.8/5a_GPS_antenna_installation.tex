\subsection{GPS Antenna and cable connection}

The PT8616 option is delivered with an active antenna and 12 meters of cable selected for the unit. Some customers however may want to use their own antenna or run longer cables. This section explains what to look out for.

\textbf{Antenna requirements and specifications.}\\
The application always requires an active antenna, running 3.3 volts. There are two types of active antennas, which may be considered. Helix or Patch. The difference are the physical design and the area of the sky which is covered. A Helix antenna has a physical shape of a pole and covers the widest area of the sky. It also has to be physically bigger, to pick up RF signals, compared to the Patch antenna. The Patch antenna can be smaller but does not cover the sky just as well as the Helix antenna. The Helix antenna may be preferred on buildings, because of the slightly better performance, but the Patch antenna suits most needs and also fits well on OB vans, roofs etc.

Different antennas have different gains which will permit different cable lengths. The typical gain level is about +30 dB. An active antenna draws current in the region of 5 - 20 mA. It is important not to draw more current than 50 mA to avoid damaging the GPS circuit. In the case of short circuit the GPS receiver shuts down the supply voltage.

\textbf{Cable loss budget.}\\
It is very important the cable loss is considered carefully when using custom cables longer than the 12 meter RG58 cable supplied. The GPS RF frequency is 1575 MHz, so all further loss calculations will be at this specific frequency. 

The GPS receiver requires a minimum signal strength of -140 dBm to lock to a satellite. The GPS satellites are specified to deliver signals strengths between -123 dBm and -130 dBm at the earth surface. With a typical antenna  gain of 30 dB, the power level out of the antenna is in the range of -97 dBm. This allows a maximum loss of 43 dB in the cable, before the locking threshold of -140 dBm is reached. It is advisable to keep a margin of about 5 dB from the locking threshold. Clouds, snow and rain will degrade the performance.
Below are some examples of cable losses:

\begin{table}[hbt]
\centering
\begin{tabular}{|l|l|l|}
\hline
\textbf{Cable type} & \textbf{Loss/100 m @ 1.5 GHz} & \textbf{Max lenght, 35 dB loss} \\ \hline
\textbf{RG58}		& ~110 dB* & 31 m  \\ \hline
\textbf{RG213}	& ~44 dB*  & 75 m  \\ \hline
\textbf{CDF400}	& 17.8 dB  & 195 m \\ \hline
\end{tabular}\\
\textit{*) Note: The RG58 and RG213 are found in various low-loss versions.\\See datasheet for specific cable used.}
\caption{Cable types.}
\label{tab:cabletypes}
\end{table}

\textbf{Antenna and cable installation and usage.}\\
The placement of the antenna is important for the overall performance. The antenna must not be obstructed in any way. This obstruction could be caused by trees, roofs etc. What may seem less obvious is tall walls near the antenna which may decrease the performance. This is because the RF-signal may reflect of the wall and the antenna could receive both the direct and reflected signals which may confuse the receiver circuit. Always install the antenna where there is a clear sky-view.

Please note it is important the cable is connected to both the antenna and the PT5300 antenna input, before the unit is powered up. The GPS receiver calculates the noise floor on power-up and the connection with the antenna has to be established at this moment.

On power up, check that the text ``GPS: none'' on the front display changes to ``GPS: ok'' after a short time. This will confirm the connection works correctly. The final step is to seal the connection to prevent corrosion, when exposed to humidity.

\textbf{Electrical requirements.}\\
\textbf{Antenna.}\\
\spectabular
\textbf{Antenna voltage:}										& 3.3 Volts\\
\textbf{Antenna maximum power consumption:}	& 50 mA \\
\textbf{Antenna minimum gain:}							& 15 dB\\
\textbf{Antenna maximum gain:}							& 50 dB\\
\textbf{Antenna maximum noise figure:}			& 1.5 dB\\
\end{tabular*}

\textbf{Receiver.}\\
\spectabular
\textbf{Receiver input level:}				& -140 dBm < \textit{input} < -5 dBm \\
\textbf{Receiver input RF frequency:}	& 1575.42 MHz \\
\end{tabular*}

\clearpage